%%
%% Arquivo principal da tese ou dissertação.
%%
%% Siga as instruções que seguem comentadas neste arquivo.
%%
%% ESTA VERSÃO REQUER UM EDITOR UTF-8.
%%
%% Exemplos de utilização dos comandos podem ser encontrados
%% nos capítulos que acompanham este pacote.
%%
%%  
\documentclass[mestrado]{ppgeq}
%\documentclass[doutorado]{ppgeq}
%\documentclass[qualify]{ppgeq}

% algumas customizações: o preamble.tex pode ser editado pelo usuário
\input{preamble}

% modifique os argumentos dos comandos abaixo para que as primeiras páginas
% sejam construídas automaticamente
\title{Modelagem e Simulação de um Reator Trifásico de Hidrogenação
Seletiva de Gasolina de Pirólise}
\author{Marcus Vinicius Pereira}
\authorCitation{Pereira, Marcus V.}
\area{Pesquisa e Desenvolvimento de Processos}
\tyear{2016}
% verificar o número de páginas do PDF final
\totalPages{103}
\advisor{Rafael de Pelegrini Soares, D.Sc.}
\advisorCitation{Soares, Rafael de P.}
\bancai{Prof. Banca 1, D.Sc.}
\bancaii{Prof. Banca 2, D.Sc.}
\bancaiii{Prof. Banca 3, D.Sc.}

\pensamento{Pensamento \\ \vspace{0.5cm} \begin{flushright}
Autor \end{flushright}
}

\agradecimento{ 
Agradecimentos
}

\resumo{
?? Adicionar aqui algumas frases com uma mini introdução/motivação ??
Neste trabalho, um reator de hidrogenação de nafta de pirólise
foi simulado, baseado nos dados publicados na literatura. O reator é do tipo leito
gotejante, e as reações consideradas são de pseudo-primeira ordem. A técnica
aplicada foi a modelagem matemática por células, onde os leitos catalíticos
foram subdivididos em reatores tipo CSTR dinâmicos associados em série. A cada
célula, um cálculo de flash foi feito, aperfeiçoando os balanços de massa e energia
comumente empregados em reatores de leito gotejante. A estimativa da fração
vaporizada, resultado do cálculo de flash, também contribui para melhorar os
cálculos hidrodinâmicos.
?? Comentar aqui o modelo termodinâmico utilizado ??
O modelo foi implementado no software \emso{}
(\emsoname{}) cerca de 9000 equações e variáveis. Os resultados foram comparados
com disponíveis na literatura.
?? Colocar as conclusões aqui ??
}
\palavraschave{1.~Gasolina de Pirólise. 2.~Hidrogenação.
3.~Reatores de leito fixo 4.~Modelagem. 5.~Simulação}

\abstract{Abstract here.}
\keywords{1.~PYGAS 2.~Hydrogenation 3.~Fixed-bed Reactors 4.~Modeling
5.~Simulation}

% Desse ponto modifique apenas a inclusão dos capítulos e apêndices
\begin{document}

% hifenizacao
\include{hyphen}

\maketitle
 

%\include{CapSymbols}
% Inclusao dos Capítulos
%
% 
%
\chapter{Introdução} \label{chap:introducao}
%\emph{Isto é uma sinopse de capítulo. A ABNT não traz nenhuma normatização a
%respeito desse tipo de resumo, que é mais comum em romances e livros técnicos.
%}

Os catalisadores tem sido aplicados pela humanidade desde a antiquidade
em atividades tais como produção de vinho, pão e queijo. Em muitos
casos, percebeu-se que a adição de uma pequena quantidade da batelada anterior
para que fosse feita a batelada atual. Mas foi em 1835 que Berzelius reuniu
obeservações feitas por químicos, sugerindo que uma pequena quantidade de
uma substância de origem externa afetavam grandemente o
curso das reações químicas. A essa substância possuidora de tal força
misteriosa Berzelius chamou de catalítica. Em $1894$, Ostwald propôs que
catalisdores são substâncias que aceleram a taxa de uma reação química, sem
serem consumidos durante a reação \cite{Oyama1988}.

A descoberta de catalisadores sólidos e suas aplicações a processos químicos no
início do século XX causou grandes avanços na industria química. A maior parte
dos processos catalíticos consiste em reatores de leito fixo. Na industria
petroquímica, reatores catalíticos de leito fixo são usados para produção de
oxido de etileno, butadieno, anidrido maleico, anidrido ftálico, estireno,
entre outros produtos. Já na industria do petróleo, destacam-se as aplicações:
reforma catalítica, isomerização, hidrodessulfurização e hidrocraqueamento
\cite{Froment2011}.

Dentre os reatores de leito fixo estão os reatores de leito gotejante
(TBR - \emph{Trickle Bed Reactors}), que nada mais são do que um tipo de reator
catalítico multifásico de leito empacotado (PBR - \emph{Packed Bed Reactor}).
Dentre as diversas aplicações dos reatores tipo PBR está a hidrogenação
seletiva de gasolina de pirólise.

A gasolina de pirólise (\emph{PYGAS}) é um dos produtos obtidos no processo de
craqueamento a vapor quando da produção de olefinas \cite{Cheng1986}. Sua
composição instável a impede de ser utilizada sem processos de refinamento
adequados \cite{Derrien1986}.

Ao longo dos anos foram desenvolvidos diversos modelos matemáticos para
entender, projetar, simular ou otimizar o desempenho de reatores, especialmente
os reatores de leito fixo. Os reatores de leito gotejante figuram entre os
reatores de leito fixo que mais exigem capacidade computacional em suas
simulações. 

Com a hidrogenação seletiva de gasolina de pirólise não
foi diferente. Vários autores se preocuparam em modelar esse
processo, adotando premissas e simplificações até hoje consideradas.
Recentemente, \citeonline{Rojas2014a} apresentaram uma modelagem de
um reator de hidrogenação seletiva de gasolina de pirólise que aplica novos
parâmetros termodinâmicos para a solubilidade de hidrogênio \cite{Rojas2014}. 

Este trabalho tem por objetivo geral o desenvolvimento de um modelo
matemático genérico para reatores de leito fixo, bifásico ou trifásicos,
aplicando a abordagem de modelagem por rede de células. Com essa
abordagem, espera-se incluir fenômenos comumente desconsiderados por
simplificações.

Como objetivo específico, pretende-se apresentar aqui a modelagem e a
simulação de um reator de hidrogenação seletiva de gasolina de pirólise,
utilizando dados publicados por \citeonline{Rojas2014a}.

Além disso, para o reator estudado neste trabalho, serão
apresentadas algumas respostas dinâmicas que refletem situações típicas da
indústria.

Para alcançar os objetivos propostos, será utilizado o simulador genérico de
processos \emso\ \cite{Soares2003} e seu ambiente de desenvolvimento de
modelos. O modelo de reator de leito fixo gerado neste estudo fará parte da
biblioteca EML (EMSO Model Library). A EML é distribuída no conceito de
\emph{software} livre, disponibilizando todos os modelos via internet e sem custo.

Esta dissertação está dividida em seis capítulos, como segue:

O \autoref{chap:introduction} (este capítulo) trata da introdução e do objetivo deste trabalho.

O \autoref{chap:revisaobibliografica} apresenta a pesquisa de referências
realizada que, inicialmente, trata de maneira breve sobre a gasolina de pirólise
e sua produção, passa pela descrição e modelagem de reatores de leito
gotejante e, finalmente, apresenta alguns aspectos da modelagem de
reatores de hidrogenação de gasolina de pirólise.

O \autoref{chap:moddesenvolvidos} descreve o processo de
hidrogenação de gasolina de pirólise e a modelagem matemática
desenvolvida deste processoo processo neste trabalho.

Os resultados da simulação do estado estacionário e de algumas
respostas dinâmicas estão no \autoref{chap:resultados}.

Por fim, no \autoref{chap:conclusoes} são apresentadas as conclusões do
trabalho bem como sugestões para trabalhos posteriores.

%
\chapter{Conceitos Fundamentais e Revisão Bibliográfica}
\label{chap:revisaobibliografica}
%\emph{Texto não obrigatório}

\section{Gasolina de Pirólise (PYGAS)} \label{sec:pygas}
A gasolina de pirólise (PYGAS - \emph{Pyrolysis Gasoline}) é um dos produtos
obtidos no processo de craqueamento a vapor quando da produção de olefinas na
indústria petroquímica. Tipicamente a gasolina de pirólise apresenta curva de
destilação entre \SI{30}{\celsius} e \SI{204}{\celsius}. Ela é um produto
instável (térmica e químicamente) devido à grande quantidade de compostos insaturados tais como
mono-olefinas, diolefinas conjugadas, estireno e outras espécies mais pesadas e
também reativas \cite{Cheng1986}.
\nomenclature[Z]{PYGAS}{Gasolina de pirólise}
 
A grande quantidade de compostos insaturados (olefinas e aromáticos) contidos
na gasolina de pirólise lhe fazem um produto interessante, tanto pela
sua alta octanagem quanto pela sua quantidade de aromáticos. Contudo, a
instabilidade (formação de goma, alteração de cor) impede sua utilização
comercial na forma como é produzida. Assim, processos de refinamento específicos
para essa corrente são necessários \cite{Derrien1986}.

O esquema de processo a ser utilizado para refinar a gasolina de pirólise
depende do produto final desejado. Para a produção de uma corrente que
comporá a gasolina automotiva, atendendo requisitos tais como estabilidade e
corrosividade, usualmente é aplicada a hidrogenação seletiva de diolefinas,
processo também conhecido como primeiro estágio de hidrogenação. Já se o
propósito for a obtenção de aromáticos, segue-se à hidrogenação seletiva o
segundo estágio de hidrogenação (hidrogenação de olefinas e hidrodessulfurização)
\cite{Derrien1986}.

\section{Reatores de Leito Gotejante} \label{sec:reatorestbr}

\subsection{Definição} \label{sec:definicao}

Na literatura há vários autores que trataram das definições de
reatores catalíticos multifásicos de leito empacotado (PBR - \emph{Packed Bed
Reactors}).
\nomenclature[Z]{PBR}{Reator de leito empacotado}
Segundo \citeonline{Froment2011}, por reatores de leito empacotado multifásicos
entende-se as colunas que possuem recheios catalíticos, e que são destinadas a
promover reações entre compostos presentes em ambas as fases gasosa e líquida.

Assim como \citeonline{Froment2011}, \citeonline{Ancheyta2011} acrescenta à
definição de PBR uma distinção quanto ao regime de escoamento e direção de
escoamento das fases, sendo duas formas: (i) em regime gotejante, onde a fase
gás é contínua e a fase líquida está distribuída, e a principal resistência à
transferência de massa está na fase gasosa; e (ii) em regime de borbulhamento,
com a fase gás distribuída e a fase líquida contínua, e a principal resistência
à transferência de massa localiza-se na fase líquida.

Os reatores de leito gotejante (TBR - \emph{Trickle Bed Reactor}) compreendem
uma família de PBRs nos quais as fases gasosa e líquida escoam através de um
leito catalítico fixo, sendo eles assim classificados como reatores de leito
fixo (FBR - \emph{Fixed Bed Reactors}). O TBR ainda possui as seguintes
características, segundo \citeonline{Ancheyta2011}:
\nomenclature[Z]{TBR}{Reator de leito gotejante}
\nomenclature[Z]{FBR}{Reator de leito fixo}

\begin{itemize}
\item A fase líquida escoa em sentido descendente (na direção da gravidade),
fluindo sob a forma de filmes, filetes (\emph{rivulet}) e gotículas (e é daí
que vem o nome gotejante (\emph{trickle}));
\item A fase gasosa pode escoar tanto ascendentemente (contracorrente à
fase líquida) quando descendentemente (cocorrente à fase líquida).
\end{itemize}

Os PBRs nos quais há escoamento ascendente, tanto da fase gasosa quanto da
fase líquida, operam em regime de borbulhamento, não sendo classificados como
um TBR \cite{Ancheyta2011}. 

\subsection {Regimes de Escoamento}
\label{sec:escoamento}

Um TBR pode ser visualizado como um leito de partículas de catalisador, no qual
o espaço intersticial entre elas formam complexos caminhos e distribuição de
poros. Ao fluir sobre as partículas de catalisador, gás e líquido podem
apresentar diferentes tipos de escoamento ou regimes. Esses regimes de
escoamento dependem da densidade do leito, da velocidade de escoamento das
fases, do tamanho das partículas de catalisador, das dimensões do reator e das
propriedades físicas dos fluidos \cite{Ranade2011}.

A definição do tipo de escoamento, tanto na fase de projeto quanto na avaliação
de um reator existente, é muito importante, pois vários parâmetros
hidrodinâmicos e de transporte são influenciados pelo regime de escoamento. Este
último influencia tanto no dimensionamento dos TBRs quanto nos
equipamentos atrelados a eles, tais como bombas e compressores
\cite{Ranade2011}.

Segundo \citeonline{Ramachandran1983}, quatro
diferentes regimes de escoamento foram identificados em TBRs:

\begin{itemize}
	\item Gotejamento (\emph{Trickle flow regime)};
	\item Pulsante (\emph{Pulse flow regime)};
	\item Spray (\emph{Spray flow regime)}; e
	\item Borbulhamento (\emph{Bubble flow regime)}.
\end{itemize}

Esses regimes de escoamento estão ilustrados na \autoref{fig:regimeescoamento}.

\begin{figure}[htb]
\centering \includegraphics[scale=0.75]{images/Chap2/regimeescoamento.png}
\caption{Regimes de escoamento em reatores de leito gotejante
\cite{Gunjal2005} (extraído de \citeonline{Ranade2011}).}
\label{fig:regimeescoamento}
\end{figure}

O escoamento em regime gotejante ocorre em baixo fluxo de líquido e moderado
fluxo de gás. Como já foi dito, o líquido escoa em forma de filmes ou filetes
sobre as partículas do catalisador. Nesse regime de escoamento, a fase contínua
é a gasosa e a fase líquida escoa dispersa. Esse tipo de escoamento também é
chamado de regime de baixa interação \cite{Saroha1996}.

Em fluxo moderado de líquido e gás ocorre o escoamento em regime pulsante,
caracterizado pelo escoamento semicontínuo de ambas as fases. Ante o regime
por gotejamento, o regime de pulso apresenta maior interação entre as fases.
Todavia, esse processo faz com que regiões ricas em gás se alternem com regiões
ricas em líquido \cite{Saroha1996}.

Os outros dois regimes de escoamento encontrados em TBRs são: borbulhamento e
tipo spray. No primeiro, a fase contínua é a líquida, enquanto que a fase gasosa
escoa dispersa; no segundo, a fase contínua é a gasosa, e a fase líquida é
dispersa. Ambos os casos são classificados como regimes de alta interação entre
as duas fases \cite{Saroha1996}.

Os reatores industriais operam com frequência próximos à transição entre os
regimes gotejante e pulsado. Isso proporciona melhores taxas de transferência de
massa, utilização do catalisador e aumenta a capacidade produtiva
\cite{Ancheyta2011}.

\subsection{Parâmetros Hidrodinâmicos}
\label{sec:parametroshidrodinamicos}

Por natureza, os TBRs possuem uma complexidade fenomenológica ímpar quando se
trata de seu comportamento hidrodinâmico. Superam em complexidade os FBRs,
onde há apenas uma fase escoando pelo leito catalítico. 

A compreensão e a previsão, ainda que imprecisamente, dos parâmetros
hidrodinâmicos dos TBRs são as chaves para o projeto e análise desse tipo de
reator. Os parâmetros sobre os quais serão discorridos a seguir são:

\begin{itemize}
  \item Perda de carga;
  \item Retenção de líquido;
  \item Molhamento das partícutas de catalisador;
  \item Coeficientes de transferência de massa;
  \item Dispersões axial e radial de massa.
%  \item Transferência de calor em TBRs
\end{itemize}

\subsubsection{Perda de Carga}
\label{sec:perdadecarga}

A capacidade de estimar a perda de carga em TBRs na fase de projeto é de suma
importância. Ela definirá a capacidade de um dos equipamentos mais custosos do
sistema, que é o compressor que fará a recirculação da fase gasosa. A perda de
carga também é um importante parâmetro a ser acompanhado pelos engenheiros quando o
TBR está em operação.

A perda de carga bifásica ao longo do reator está relacionada com: (i) a
geometria do reator (diâmetro, tamanho e forma do catalisador e geometria dos
internos, tais como prato distribuidor); (ii) parâmetros operacionais tais como
vazão de gás e líquido (regime de escoamento); e (iii) propriedades das fases
(densidade, viscosidade, tensão superficial etc.). Temperatura e pressão de operação afetam
indiretamente a perda de carga através das propriedades do fluido
\cite{Ranade2011}.

Normalmente as equações de perda de carga são definidas em termos de perda de
carga específica $\Delta P/L$, que é definida como a variação da pressão interna
por unidade de comprimento do reator. Em reatores cujo regime é de baixa
interação (regime gotejante de escoamento), a perda de carga tende a ser
pequena. Para regimes de alta interação, a perda de carga pode chegar a algumas
atmosferas por metro \cite{Benkrid1997}.

\nomenclature[G]{$\Delta P/L$}{Perda de carga por unidade de comprimento de
reator \nomunit{kPa/m}}

Segundo \citeonline{Holub1993}, os modelos hidrodinâmicos podem ser
classificados em duas categorias. A primeira categoria lança mão do empirismo
baseado em análise dimensional para produzir correlações de perda de carga e
retenção de líquido. A segunda categoria utiliza equações do tipo Ergun
\apud{Ergun1952}{Holub1993}, modificando parâmetros para escoamento bifásico.
Essa abordagem é aplicada especialmente para regimes de escoamento de baixa
interação.

\citeonline{Holub1993}, por exemplo, desenvolveram uma correlação para perda
de carga para TBRs operando em regimes de baixa interação. Já
\citeonline{Benkrid1997} utilizaram dados experimentais  da literatura para
propor modelos simples de perda de carga em TBRs que operam em regimes de alta
interação. Ambos os autores citados também desenvolveram
correlações para retenção de líquido.

Diferentemente de muitos autores, que propuseram equações utilizando dados
experimentais coletados em pressões atmosféricas, \citeonline{Larachi1991}
construíram correlações tanto para perda de carga quanto para retenção de
líquido, em pressões de até \SI{8,1}{MPa}.

\subsubsection{Retenção de Líquido}
\label{sec:retencaodeliquido}

A retenção de líquido (\emph{liquid holdup}) em um leito de TBR  pode ser
expressada de duas maneiras: (i) retenção total ($\epsilon^L$), definida como o
volume de líquido por unidade de volume de leito e (ii) saturação de líquido
($\beta_L$), definida como o volume de líquido por unidade de volume vazio (ao
invés de volume total de leito).

A retenção de líquido é composta de duas partes: estática ($\epsilon_{Ls}$) e
dinâmica ($\epsilon_{Ld}$). Por retenção de líquido estática entende-se a
porção de líquido que permanece na superfície das partículas de catalisador após o
leito ser drenado. A fração removida de líquido no processo de drenagem é
definida como retenção dinâmica \cite{Ranade2011}. Alguns autores utilizam como
referência a saturação de líquido estática ($\beta_{Ls}$) e saturação de líquido
dinâmica ($\beta_{Ld}$).

Há algumas ténicas para determinação da retenção de líquido em TBRs. De acordo
com \citeonline{Benkrid1997}, a técnica mais comum é a da drenagem, já
mencionada. Essa técnica consiste em cessar simultaneamente a alimentação de gás
e líquido no reator (topo), coletando o líquido na saída (fundo). O volume de
líquido drenado significará diretamente $\epsilon_{Ld}$. Outra técnica consiste
em determinar a massa do reator enquanto seco e em operação, quando gás e
líquido estarão escoando pelo leito catalítico. Esse método levará diretamente à
determinação de $\beta_L$. \citeonline{Larachi1991} utilizam a distribuição do
tempo de residência para chegar a $\beta_L$.

Vários fatores em TBRs são dependentes da retenção de líquido, tais como fator
de molhabilidade e coeficientes de transferência de massa e calor. A retenção de
líquido também determina o tempo de residência de líquido dentro do reator e,
portanto, a conversão dos reagentes \cite{Ranade2011}.

A consequência das definições apresentadas nessa seção podem ser sumarizadas
da seguinte forma:
\begin{equation}
\epsilon^L + \epsilon^G = \epsilon_B
\label{eq:balancoespaco1}
\end{equation}
sendo $\epsilon_B$ a porosidade do leito e $\epsilon^G$ a retenção de gás.

\nomenclature[G]{$\epsilon^{L}$}{Retenção total de líquido}
\nomenclature[G]{$\epsilon_{Ls}$}{Retenção estática de líquido}
\nomenclature[G]{$\epsilon_{Ld}$}{Retenção dinâmica de líquido}
\nomenclature[G]{$\epsilon_{B}$}{Porosidade do leito}
\nomenclature[G]{$\beta_L$}{Saturação de líquido}
\nomenclature[G]{$\beta_{Ls}$}{Saturação de líquido estática}
\nomenclature[G]{$\beta_{Ld}$}{Saturação de líquido dinâmica}

\subsubsection{Molhamento das Partículas de Catalisador}
\label{sec:molhamento}

Dentre os diversos tipos de reatores multifásicos, o molhamento das partículas
de catalisador é um fenômeno exclusivo em TBR. Levando-se em conta que em um
TBR a fase líquida escoa de forma não uniforme, a determinação do molhamento do
catalisador é uma tarefa bastante difícil \cite{Ranade2011}

Na literatura são definidos dois tipos de molhamento. \citeonline{Colombo1976}
os descrevem assim:

\begin{itemize}
	\item Molhamento interno ou enchimento de poro: é o volume de líquido dentro
	dos poros do catalisador. Como as partículas de catalisador são quase sempre
	porosas, o molhamento interno é considerado total devido ao efeito de
	capilaridade;
	\item Molhamento externo: é a área externa à partícula de catalisador
	efetivamente em contato com o líquido que escoa. Praticamente toda a
	transferência de massa entre o líquido presente nos poros do catalisador e o
	líquido que escoa por sobre as partículas ocorre nessa área.
\end{itemize}

Isto posto, a \emph{eficiência de molhamento} ($\eta_{CE}$) deriva da definição
de molhamento externo. Em outras palavras, ela é definida como sendo a
porcentagem da área superficial externa do catalisador que está efetivamente em
contato com o líquido que escoa \cite{Schwartz1976}.

De acordo com a literatura disponível, a medição da eficiência de molhamento
pode ser direta e indireta. Entre os métodos diretos estão técnicas
fotográficas, método de injeção de corante, imagem de ressonância magnética,
entre outros \apud{Sederman2001}{Ranade2011}. Já os métodos indiretos são
aplicáveis a unidades industriais além de possuirem melhor custo-benefício
\cite{Ranade2011}. \citeonline{Colombo1976} e \citeonline{Schwartz1976}, por
exemplo, utilizaram métodos baseados tanto em taxa de reação quanto em
transferência de massa para determinar a eficiência de molhamento.

\nomenclature[G]{$\eta_{CE}$}{Eficiência de molhamento}

\subsubsection{Coeficientes de Transferência de Massa}
\label{sec:coeficientes}

Em reatores com escoamento gotejante não há um mecanismo severo de mistura como
em outros tipos de escoamento (tanques de mistura ou regimes de alta interação).
Logo, as taxas de transferência de massa são menores em TBRs do que em outros
reatores, podendo limitar a taxa das reações. Os três tipos de taxas de
transferência de massa relevantes nos TBRs são: gás-líquido, líquido-sólido e
gás-sólido \cite{Ranade2011}.

A ocorrência desses tipos de transferência de massa dependerá da eficiência de
molhamento e da retenção de líquido. Em reatores com eficiência de molhamento
próxima à totalidade, por exemplo, desconsidera-se a transferência de massa
gás-sólido \cite{Ranade2011}.

\subsubsection{Dispersões Axial e Radial de Massa}
\label{sec:dispersoes}

As dispersões axial e radial são dois fenômenos que ocorrem em TBRs,
desviando-os da idealidade de escoamento contida no conceito de PFR (PFR -
\emph{Plug-Flow Reactor}).

\nomenclature[Z]{PFR}{Reator de escoamento empistonado}

A dispersão radial está diretamente ligada à distribuição de líquido na entrada
do reator. Esse efeito possui uma quantidade pequena de estudos na literatura. A
razão entre o diâmetro do reator e o diâmetro da partícula de catalisador
($D_R/d_p$) influenciam na relevância da distribuição radial de massa
\cite{Saroha1998}. \citeonline{Al-Dahhan1994}, por exemplo, sugerem que, mesmo
para reatores em alta pressão, um valor de $D_R/d_p > 20$ minimiza o efeito de
má distribuição de líquido em TBR.

Diferente do que ocorre com a dispersão radial, a dispersão axial é um fenômeno
amplamente estudado, com várias referências na literatura. Esse fenômeno está
ligado a retromistura de líquido dentro do leito catalítico e está relacionado
com a razão entre o comprimento do reator e diâmetro da partícula de catalisador
($L_R/d_p$) \cite{Ancheyta2011}. \cite{Mears1971b} propôs um critérios para
verificar se a dispersão axial é relevante ou não, baseando-se no número de
Peclet ($Pe$). O critério de Mears é apresentado da seguinte maneira:

\begin{equation}
\dfrac{L_{R}}{d_p} > \dfrac{20.n}{Bo}ln\left(\dfrac{C_o}{C_f}\right)
\label{eq:Mearscriterio}
\end{equation}

\begin{equation}
Bo = \dfrac{ud_p}{D}
\label{eq:Bodenstein}
\end{equation}

 \begin{equation}
 Pe = \dfrac{uL_{R}}{D}
 \label{eq:Peclet}
 \end{equation}
onde $C_o$ é a concentração do reagente na entrada do reator, $C_f$ é a
concentração do reagente na saída do reator, $L$ é o comprimento do reator, $n$
é a ordem da reação e $Bo$ (número Bodenstein) é o número de $Pe$ baseado no
diâmetro da partícula.

\nomenclature{$D_{R}$}{Diâmetro do reator \nomunit{m}}
\nomenclature{$L_{R}$}{Comprimento total do reator \nomunit{m}}
\nomenclature{$d_{p}$}{Diâmetro da partícula de catalisador \nomunit{m}}
\nomenclature{$Bo$}{Número de Bodenstein}

Alguns fatores que contribuem na distribuição de líquido e na retromistura são:
capilaridade, existência de zonas-mortas, molhamento parcial e caminhos
preferenciais \cite{Ranade2011}.

\subsection{Reações Químicas em TBR} \label{sec:reacoestbr}
 
As reações químicas em TBRs possuem uma complexidade maior ante as reações
homogêneas justamente pela presença de um catalisador sólido. A superfície
de uma partícula de catalisador, em toda a sua complexidade, apresenta-se
como o local onde as reações ocorrem. Além disso, fenômenos de transporte
de massa podem influenciar na taxa global da reação \cite{Froment2011}. 

Sendo assim, os passos envolvidos em uma reação química, na superfície de um
catalisador, para um reator cujo escoamento é monofásico são \cite{Froment2011}: 

\begin{enumerate}
\item Transporte dos reagentes do seio do fluido para a superfície
do catalisador;
\item Transporte dos reagentes nos poros do catalisador;
\item Adsorção dos reagentes no sítio catalítico;
\item Reação química entre os átomos ou moléculas;
\item Dessorção dos produtos;
\item Transporte dos produtos da reação dos poros de volta à
superfície do catalisador;
\item Transporte dos produtos da superfície das partículas para o seio do
fluido.
\end{enumerate} 

Segundo \apudonline{Ramachandran1983}{Ranade2011}, para os TBRs onde as
partículas de catalisador estão completamente molhadas (velocidade superficial
de líquido \SI{>0,5}{\centi\meter\per\s}), os passos apontados por
\citeonline{Froment2011} ganham novas etapas, como segue:

\begin{enumerate}
\item Transporte dos reagentes presentes na fase gasosa para o seio da
fase líquida;
\item Transporte dos reagentes presentes na fase líquida para a superfície do
catalisador;
\item Transporte dos reagentes nos poros do catalisador;
\item Adsorção dos reagentes no sítio catalítico;
\item Reação química entre os átomos ou moléculas;
\item Dessorção dos produtos;
\item Transporte dos produtos da reação dos poros de volta à superfície do
catalisador;
\item Transporte dos produtos da superfície das partículas para o seio da fase
líquida;
\end{enumerate} 

Pode haver ainda mais um passo em que um dos produtos, se termodinamicamente
favorável, passasse para a fase gasosa. Em reatores de hidrotratamento de
diesel, por exemplo, o \ce{H2S} produzido descola-se parcialmente para a fase
gasosa, diminuindo a pressão parcial de hidrogênio. O hidrogênio nesse caso é
o reagente em excesso \cite{Ancheyta2011}.

Entretanto, muitos TBRs comerciais operam com velocidades superficiais de
líquido baixas \SI{<0,5}{\centi\meter\per\s}. Isso faz com que o regime de
escoamento seja classificado como de baixa interação e, consequentemente, não molha
completamente a superfície das partículas. Nesse caso, a taxa de reação
dependerá da fase em que está o reagente limitante \cite{Ranade2011}.

\section {Modelagem de Reatores de Leito Fixo} \label{sec:modelagemreatores}

Várias formas de modelagem de FBRs podem ser encontradas na literatura.
Segundo \citeonline{Froment2011}, os modelos podem ser categorizados em dois
grandes grupos: pseudo-homogêneos e heterogêneos. Por modelos
pseudo-homogêneos entende-se aqueles que não fazem distinção entre as fases,
i.e., não considera a presença do catalisador. Em se tratando de um TBR, não há
também distinção entre as fases líquida e gasosa. Já modelos heterogêneos
distinguem, no mínimo, duas fases concomitantes.

Ainda de acordo com \citeonline{Froment2011}, os FBR podem ser classificados em
unidimensionais e bidimensionais. Os modelos bidimensionais consideram
gradientes radiais de massa e/ou calor. Já modelos unidimensionais consideram a
idealidade de escoamento empistonado, podendo ou não contemplar a dispersão
axial.

Modelos matemáticos também podem ser categorizados ou como de
descrições empíricas (também chamados de modelos caixa-preta) ou que se baseiam
em primeiros princípios. Como nome diz, modelos de primeiros princípios
utilizam-se das leis químicas e físicas (balanço de massa e energia,
termodinâmica, cinética das reações). Por outro lado, modelos empíricos são
utilizados quanto há falta de tempo e/ou informações para o
desenvolvimento de modelos de primeiros princípios. Nesses modelos, são
necessárias informações de entrada e saída para ajustar coefientes nas equações
que os compõem \cite{Edgar2001}.

De forma genérica, \citeonline{Rasmuson2014} classificam modelos matemáticos
em pares opostos, por exemplo: determinísticos versus estocásticos; contínuos
versus discretos. 

De acordo com \citeonline{Rasmuson2014}, os modelos determinísticos são aqueles
em que a cada variável e parâmetro pode ser atribuído um número fixo definido,
ou uma série de números, para um dado conjunto de condições, i.e., o modelo não
possui componentes que sejam incertos. Em contraste, a incerteza é
introduzida nos modelos estocásticos. Neste tipo de modelo, 
as variáveis ou parâmetros utilizados para descrever as relações de
entrada-saída não são precisamente conhecidas. Além disso, Um modelo estocástico
envolve parâmetros caracterizadas por distribuições de probabilidade. Por essa
razão, tais modelos podem apresentar diferentes resultados em cada simulação.

Segundo \citeonline{Rasmuson2014}, em modelos contínuos as variáveis
podem assumir qualquer valor dentro de um intervalo definido. Em se tratando de
PBR, \citeonline{Ancheyta2011} modelos contínuos contêm um conjunto de equações
diferenciais com uma ou mais variáveis independentes, onde todas devem ser
resolvidas simultaneamente. Em contrapartida, modelos discretos consideram que as
variáveis assumem apenas um valor dentro de um determinado intervalo
\cite{Rasmuson2014}. Diferentemente dos modelos contínuos (equações
diferenciais), a modelagem discreta encarrega-se de resolver equações algébricas
para o caso de estado estacionário, ou equações diferenciais ordinárias quando
se trata de simulação dinâmica \cite{Schnitzlein1987}.

\citeonline{Deans1960} são apontados como os primeiros a construir um modelo
baseado em estágios discretos em TBRs, chamado de \emph{rede de células} por
alguns autores (\citeonline{Schnitzlein1987} e \citeonline{Ancheyta2011}, por
exemplo). \citeonline{Deans1960} utilizaram o método de modelagem por rede
de células para estudar os fenômenos de dispersão axial e radial em um TBR.
Segundo \citeonline{Schnitzlein1987}, independente do arranjo das células, a
descrição do sistema reacional dentro de cada célula pode ser pseudo-homogêneo ou
heterogêneo.

Outra abordagem de modelagem discreta são os modelos \emph{baseados em
estágios}. Nesses modelos utiliza-se \emph{abordagem baseada em taxa}, que é
fundamentada na aplicação direta dos fenômenos de difusão, transferência de
calor e dos efeitos de interação multicomponente no cálculo de cada segmento
(estágio). Essa abordagem é bastante utilizada em processos de separação tais
como destilação e extração \cite{Jakobsson2004}. Foi com essa abordagem que
\citeonline{Jakobsson2004} modelaram reatores de hidrotratamento de diesel, em
suas operações contracorrente e cocorrente.

\section {Modelagem e Simulação de Hidrogenação Seletiva de Gasolina de Pirólise}
\label{sec:hidrogenacaopygas}

O primeiro estágio de hidrogenação da gasolina de pirólise, também conhecido
como hidrogenação seletiva, foi estudado por alguns autores encontrados na
literatura, incluindo algumas avaliações de resposta dinâmica. A seguir serão
apresentados alguns estudos publicados.

\subsection{Descrição Geral do Processo} \label{sec:descricaogeral}

Tipicamente, a gasolina de pirólise é hidrogenada em um reator tipo TBR, onde o
catalisador pode ser de paládio ou níquel suportado em alumina. Normalmente, o
reator possui ao menos dois leitos. Devido à exotermicidade das reações, há uma
injeção de \emph{quench} líquido entre os leitos, composta de produto
hidrogenado resfriado, que serve para controlar a temperatura do segundo leito e
evitar o coqueamento sobre a superfície do catalisador. O produto hidrogenado
pode também diluir a carga do reator, retardando o processo de desativação já
mencionado \cite{Cheng1986,Derrien1986,Arpornwichanop2002,Rojas2014a}.

\subsection{Cinética das Reações} \label{sec:cinetica}

As duas principais reações que ocorrem em reatores de hidrogenação de gasolina
de pirólise são a conversão de diolefinas em olefinas e de olefinas em
parafinas, que ocorrem de forma sequencial. Como já mencionado, o grau de
hidrogenação dependerá da utilização final da corrente hidrogenada
\cite{Cheng1986}.

\citeonline{Cheng1986} publicaram valores de constantes cinéticas para a
hidrogenação de gasolina de pirólise, agrupando os compostos em diolefinas,
olefinas e parafinas. Além disso, as reações foram tratadas como irreversíveis,
da seguinte forma:
\begin{equation}
\textrm{Diolefinas} \rightarrow \textrm{Olefinas} \rightarrow \textrm{Parafinas}
\label{eq:hidrogenacao}
\end{equation}

Enfim, \citeonline{Cheng1986} estimaram as constantes cinéticas assumindo que
as reações fossem de pseudo-primeira ordem, ou seja, a concentração de
hidrogênio na fase líquida foi assumida como constante.

\citeonline{Authayanun2008} apresentam uma simulação de um reator de
hidrogenação parcial de gasolina de pirólise. Um dos objetivos deste trabalho
foi a determinação das constantes de velocidade específicas, assumindo as
mesmas considerações feitas por \citeonline{Cheng1986}, exceto que as
reações fossem de pseudo-primeira ordem, \emph{i.e.}, a concentração de
hidrogênio compôs a equação da velocidade das reações. Além disso, foi
adicionado um termo de adsorção à cinética de reação das olefinas, uma vez que a
reação destas é fortemente influenciada pela concentração de diolefinas, como
demonstrado recentemente por \citeonline{Zhou2010}.

\citeonline{Rojas2014a} e \citeonline{Mostoufi2005a} utilizaram praticamente as
mesmas premissas apresentadas por \citeonline{Cheng1986}, porém com uma
diferença: os compostos insaturados não foram agrupados em diolefinas
(pentadieno e ciclopentadieno, por exemplo) ou olefinas (penteno e
ciclopenteno). As cinéticas foram específicas para cada composto reagente.
\citeonline{Hanika1999} apresentam as reações e os parâmetros cinéticos
utilizados por \citeonline{Rojas2014a} e \citeonline{Mostoufi2005a}.

\subsection{Considerações sobre Propriedades Físicas e Equilíbrio Líquido-Vapor}
\label{sec:consideracoeselv}

Como já mencionado na \autoref{sec:modelagemreatores}, modelos de reatores
determinísticos contínuos necessitam resolver um sistema de equações
diferenciais para obter os resultados de simulação. Nesses modelos, a inserção
de equações que descrevam o equilíbrio líquido-vapor (ELV) é trabalhoso. Em
TBRs de hidrotratamento de diesel, segundo \citeonline{Ancheyta2011}, as
premissas de (i) fase gasosa em excesso, cuja composição é próxima a de
hidrogênio puro, e (ii) de que os compostos em fase líquida não mudam de fase,
facilitam os cálculos por essa abordagem.

Nos estudos realizados por \citeonline{Authayanun2008},
\citeonline{Arpornwichanop2002}, \citeonline{Arpornwichanop2008} e
\citeonline{Mostoufi2005a} as propriedades das fases gasosa e líquida foram
mantidas constantes ao longo de todo o reator, inclusive a concentração de
hidrogênio na fase líquida. A consequência disso é que, nesses trabalhos, os
efeitos de aquecimento tanto na taxa da reação quanto nos fenômenos
hidrodinâmicos permanecem desconhecidos.

Entretanto, \citeonline{Rojas2014a}, em sua modelagem de um reator de
gasolina de pirólise, modelam o ELV através de um \emph{flash} adiabático,
utilizando a abordagem $\phi-\phi$, onde uma equação de estado é utilizada para
o cálculo de equilíbrio de fases. Isso foi feito porque um dos objetivos do
trabalho foi comparar diferentes equações de estado (e seus parâmetros) quanto a
solubilidade de hidrogênio em gasolina de pirólise; a base termodinâmica
utilizada está no trabalho de \citeonline{Rojas2014}, que é uma extensão do
estudo feito por \citeonline{Zhou2006}.
\nomenclature[Z]{ELV}{Equilíbrio Líquido-Vapor}

\subsection{Hidrodinâmica} \label{sec:hidrodinamica}
Todos os trabalhos encontrados sobre hidrogenação parcial de gasolina de
pirólise classificam o reator como um TBR \cite{Arpornwichanop2002,
Authayanun2008, Mostoufi2005a, Arpornwichanop2008, Rojas2014a}. 

O trabalho de \citeonline{Rojas2014a} destaca-se dos demais autores
citados por considerar o regime de escoamento do reator como sendo do tipo
borbulhamento, ainda que ambas as fases estejam em escoamento descendente.
Assim, algumas equações hidrodinâmicas consideradas são diferentes das
utilizadas pelos demais trabalhos.

\subsection{Transferência de Massa} \label{sec:transferenciamassa}

Sobre o fenômeno de transferência de massa do sistema difusão-reação, 
os autores adotaram posturas conforme as respectivas abordagens cinéticas.

Assim, o grupo de autores que agrupou os compostos em diolefinas, olefinas e
parafinas \cite{Arpornwichanop2002, Authayanun2008} e, ainda, que considerou o
hidrogênio na cinética da reação, também tratou da difusão deste composto da
fase gasosa até a superfície do catalisador. 

Em contrapartida, \citeonline{Rojas2014a} e \citeonline{Mostoufi2005a}, por
terem utilizado como base o trabalho de \citeonline{Hanika1999} para a cinética
das reações, consideraram a fase gasosa em grande excesso e, portanto,
neglicenciaram a resistência à transferência de massa na fase gasosa.
Logo, segundo esses autores, a resistência à transferência de massa
mais influente está na fase líquida (interface líquido-sólido).

\subsection{Abordagens de Modelagem} \label{sec:abordagensmodelagem}

Os pesquisadores que publicaram trabalhos sobre hidrogenação de gasolina
de pirólise, seja a hidrogenação parcial, seja a hidrogenação total, modelaram
os reatores utilizando modelos determinísticos contínuos. A diferença entre
eles fica a cargo do método númerico aplicado. Vale destacar que no caso de
\citeonline{Rojas2014a}, foi elaborado um algoritmo para a resolução
do reator, e que ainda utilizou de simulador comercial para o cálculo de ELV.

Não foi encontrado qualquer artigo que utilizasse a modelagem por rede de
células para o processo de hidrogenação de gasolina de pirólise.

%\section{Considerações Finais} \label{sec:consideracoesfinais}

%
\chapter{Metodologia}
\label{chap:metodologia}

\section{Descrição do Processo} \label{sec:descricaoprocesso}

Como já mencionado no \autoref{chap:introducao}, o processo modelado
neste trabalho utilizou os dados publicados por \citeonline{Rojas2014a}. 

\subsection{Diagrama Esquemático} \label{sec:diagramaesquematico}

O reator é composto de dois leitos catalíticos, carregados de forma densa
(\emph{dense loading}). No topo de ambos os leitos há catalisador carregado de
forma solta (\emph{sock loading}). A carga é composta de gasolina de pirólise,
hidrogênio e parte do produto hidrogenado reciclado. Este último é parte da
corrente líquida oriunda de um vaso de separação, cuja carga é a corrente de
saída do reator. Parte da corrente líquida do vaso de separação também
funciona como corrente de resfriamento, que é injetada entre os dois leitos
catalíticos para controle de temperatura (\emph{quench}). Um diagrama
simplificado que ilustra o reator está na \autoref{fig:esquemareator}. 

\begin{figure}[htb]
\centering \includegraphics[scale=0.75]{images/Chap3/esquemareator.png}
\caption{Diagrama esquemático do reator \cite{Rojas2014a}.}
\label{fig:esquemareator}
\end{figure}

A \autoref{tab:dadosreator} apresenta as dimensões do reator.

\begin{table}[!htb]
\begin{center}
\caption{Dados do Reator \cite{Rojas2014a}.}
\label{tab:dadosreator}
\small
\begin{tabular}{lcc}
{Dimensão} & {Variável} & {Valor}
\\
\hline
{Diâmetro do Reator} & {$D_R$} & $3,047$ m \\
{Comprimento do Leito Solto 1} & {$L_{sl1}$} & $0,13$ m \\
{Comprimento do Leito Denso 1} & {$L_{dl1}$} & $2,97$ m \\
{Comprimento do Leito Solto 2} & {$L_{sl2}$} & $0,38$ m \\
{Comprimento Leito Denso 2} & {$L_{dl2}$} & $2,97$ m \\
{Zona de Quench} & {$L_{Q}$} & $1,55$ m \\
\bottomrule
\end{tabular}
\end{center}
\end{table}

\nomenclature{$L_{sl1}$}{Comprimento do leito solto 1 \nomunit{m}}
\nomenclature{$L_{dl1}$}{Comprimento do leito denso 1 \nomunit{m}}
\nomenclature{$L_{sl2}$}{Comprimento do leito solto 2 \nomunit{m}}
\nomenclature{$L_{dl2}$}{Comprimento do leito denso 2 \nomunit{m}}
\nomenclature{$L_{Q}$}{Comprimento da zona de quench \nomunit{m}}

\subsection{Catalisador} \label{sec:catalisador}

O catalisador considerado na modelagem contém Pd suportado em alumina
($Al_2O_3$), de formato esférico. Além desses dados, foi fornecida a informação
de volume total de poros $Vt_{poros}$ = $5,5.10^{-4}$ $m^3/kg$. O dado de volume
total de poros não é suficiente por si só para calcular a porosidade dos leitos,
como é apresentado na \autoref{sec:propriedadesleitoscataliticos}. Assim,
adotou-se a massa específica do $Al_2O_3$ puro ($\rho_{Al_2O_3}$ = $3940$
$kg/m^3$) para a massa específica da parte sólida das partículas de catalisador.

\nomenclature{$Vt_{poros}$}{Volume total de poros \nomunit{m^3/kg}}
\nomenclature[G]{$\rho_{Al_2O_3}$}{Massa específica do óxido de alumínio
\nomunit{kg/m^3}}

\subsection{Porosidade dos Leitos Catalíticos}
\label{sec:propriedadesleitoscataliticos}

Antes de determinar a porosidade dos leitos catalíticos, é preciso avaliar a
massa específica das partículas ($\rho_{p}$). Este parâmetro, calculado pela
\autoref{eq:rhoparticula}, é estimado em $1244 kg/m^3$.

\begin{equation}
\rho_{p} = \dfrac{1}{Vt_{poros}+\dfrac{1}{\rho_{Al_2O_3}}}
\label{eq:rhoparticula}
\end{equation}

\nomenclature[G]{$\rho_{p}$}{Massa específica da partícula de catalisador
\nomunit{kg/m^3}}

A porosidade de cada um dos leitos pode ser calculada, portanto, pela
\autoref{eq:porosidade}, como segue:

\begin{equation}
\epsilon_{B} = \dfrac{V_{l}-\dfrac{\rho_{B}V_l}{\rho_{Al_2O_3}}}{V_l}
\label{eq:porosidade}
\end{equation}

sendo $V_l$ o volume do leito e $\rho_B$ a
massa específica de um leito catalítico logo após o seu carregamento. 

Utilizando os dados apresentados na \autoref{sec:catalisador}, e utilizando os
valores de massa específica \emph{bulk} dos leitos $\rho_{B}$ publicados por
\citeonline{Rojas2014a}, chega-se nos valores mostrados na
\autoref{tab:dadosdosleitos}.

\begin{table}[!htb]
\begin{center}
\caption{Dados dos Leitos Catalíticos \cite{Rojas2014a}.}
\label{tab:dadosdosleitos}
\small
\begin{tabular}{ccc}
{ - } & {Densidade \emph{bulk} ($kg/m^3$)} & {Porosidade} 
\\
\hline
{Leito 1} & $814$ & $0,346$ \\
{Leito 2} & $813$ & $0,347$ \\
\bottomrule
\end{tabular}
\end{center}
\end{table}

Visto que o topo dos leitos, carregados de maneira solta, não são partes
significativas dos leitos, a porosidade destas seções foi
considerada a mesma de carregamento denso para fins de simplificação. 

\subsection{Composição das Correntes} \label{sec:composicaocorrentes}

A composição das correntes de entrada e de quench estão na
\autoref{tab:composicao}. Para o presente trabalho, utilizou-se
apenas os valores da primeira corrida publicados por
\citeonline{Rojas2014a}, já que para este caso foram publicados
também dados industriais.

As propriedades termodinâmicas dos compostos $28$ e $29$ não foram
encontradas. Assim, eles foram considerados como sendo parte dos compostos
$26$ e $27$, respectivamente.

\begin{table}[!htb]
\begin{center}
\caption{Composição das correntes de entrada e de quench \cite{Rojas2014a}.}
\label{tab:composicao}
\small
\begin{tabular}{clcc}
{Identificador $i$} & {Composto} & Entrada (\% mássica) & Quench (\% mássica)
\\
\hline
1 & Hidrogênio				& $0,48$ & $0,08$ \\
2 & Metano					& $0,52$ & $0,70$ \\
3 & Etano					& $0,12$ & $0,11$ \\
4 & n-Propano				& $0,36$ & $0,27$ \\
5 & n-Butano				& $0,30$ & $0,24$ \\
6 & n-Pentano				& $5,40$ & $5,60$ \\
7 & trans-2-Penteno			& $5,30$ & $7,60$\\
8 & trans-1,3-Pentadieno	& $2,50$ & $0,23$ \\
9 & Ciclopentano			& $1,50$ & $2,60$ \\
10& Ciclopenteno			& $2,10$ & $3,00$ \\
11& Metil-1,3-Ciclopentadieno	& $1,90$ & $0,21$ \\
12& n-Hexano				& $3,30$ & $3,30$ \\
13& Metilciclopentano		& $1,60$ & $1,70$ \\
14& Metilciclopenteno		& $2,00$ & $2,60$ \\
15& 1,3-Ciclopentadieno		& $1,90$ & $0,02$ \\
16& Benzeno					& $28,90$ & $30,10$ \\
17& n-Heptano				& $2,80$ & $2,90$ \\
18& Tolueno					& $16,00$ & $16,30$ \\
19& n-Octano				& $1,30$ & $1,30$ \\
20& Etilbenzeno				& $3,40$ & $5,40$ \\
21& Estireno				& $2,00$ & $0,11$ \\
22& Xileno					& $5,40$ & $5,50$ \\
23& n-Nonano				& $0,66$ & $0,72$ \\
24& 1-Metil-3-Etilbenzeno	& $2,90$ & $3,90$ \\
25& Metilestireno			& $1,40$ & $0,31$ \\
26& Dihidrodiciclopentadieno	& $1,80$ & $3,00$ \\
27& Diciclopentadieno		& $1,80$ & $0,12$ \\
28& Metildihidrodiciclopentadieno	& $1,50$ & $2,20$ \\
29& Metildiciclopentadieno	& $0,88$ & $0,11$ \\
\bottomrule
\end{tabular}
\end{center}
Onde $i$ é o número identificador do composto na simulação a ser apresentada.
\end{table}

\nomenclature[S]{$i$}{i-ésimo componente}

\subsection{Condições de operação} \label{sec:condicaocomposicaocorrentes}

As condições de operação consideradas inicialmente estão na
\autoref{tab:condicoesoperacao}. No \autoref{chap:resultados} há uma discussão
sobre a validade de algumas dessas variáveis aqui apresentadas.

\begin{table}[!htb]
\begin{center}
\caption{Dados do Reator \cite{Rojas2014a}.}
\label{tab:condicoesoperacao}
\small
\begin{tabular}{lcc}
{Dimensão} & {Variável Discretizada} & {Valor}
\\
\hline
{Pressão de entrada} & {$P_{1}$} & $5,03$ MPa \\
{Temperatura de entrada} & {$T_{1}$} & $366,15$ K \\
{Vazão mássica de entrada} & {$F_{w,1}$} & $2,477.10^5$ kg/h \\
{Pressão da corrente de \emph{quench}} & {$P_{Q}$} & $4.6$ MPa \\
{Temperatura da corrente de \emph{quench}} & {$T_{Q}$} & $326,15$ K \\
{Vazão mássica da corrente de quench} & {$F_{w,Q}$} & $2,079.10^4$ kg/h \\
\bottomrule
\end{tabular}
\end{center}
\end{table}

\nomenclature{$F_w$}{Vazão mássica \nomunit{kg/h}}
\nomenclature[S]{$w$}{Referente à massa}

\section{Premissas} \label{sec:premissas}

\subsection{Premissas adotadas por \citeonline{Rojas2014a}}
\label{sec:premissasrojas}

Antes de apresentar as premissas que nortearam o presente trabalho, vale
aqui apresentar, para efeito de comparação, as premissas utilizadas por
\citeonline{Rojas2014a}.

\begin{enumerate}
  \item O reator opera em estado estacionário e adiabaticamente.
  \item Gradientes radiais são desprezíveis.
  \item Dispersões axiais foram negligenciadas; portanto, assumiu-se o
  escoamento empistonado para ambas as fases líquida e gasosa.
  \item A fase gasosa está em excesso; dessa forma, negligenciou-se a
  resistência a transferência de massa na fase gasosa.
  \item Fator de molhamento, atividade catalítica e densidade do leito
  uniformes.
  \item A transferência de calor entre as fases e no interior das partículas de
  catalisador foram desprezadas.
  \item A entalpia de dissolução de compostos na fase gás, bem como calor de
  vaposização de líquido foram desprezadas.
  \item A região do quench é assumida como sendo um tambor \emph{flash}, que atinge o
  equilíbrio instantaneamente.
  \item A corrente de saída do primeiro leito mistura-se instantaneamente com a
  corrente de quench.
  \item O reator opera em regime de borbulhamento.
  \item Reações reversíveis e isomerizações foram negligenciadas.
  \item As reações ocorrem somente na interface líquido-sólido.
  \item A desativação catalítica foi desprezada.
\end{enumerate}

No presente estudo, algumas premissas adotadas por \citeonline{Rojas2014a} serão
alteradas, como será descrito a seguir. 

\subsection{Premissas deste trabalho} \label{sec:premissasdestetrabalho}

Como já mencionado no \autoref{chap:introducao}, um dos objetivos do presente
trabalho é avaliar algumas respostas dinâmicas ao processo proposto. Além disso,
a implementação da modelagem foi feita em \emso, software que avalia respostas
dinâmicas por concepção. Assim, convenientemente, a premissa de um reator
operando em estado estacionário foi descartada. Permanece, porém, aqui, a
premissa de que o equipamento opere de forma adiabática.

A alteração mais importante às premissas adotadas por \citeonline{Rojas2014a}, e
que representa um avaço por eles publicado, é a consideração tanto da entalpia
de dissolução do hidrogênio na fase líquida quanto da entalpida de vaporização
de líquido. Isso foi possível graças à abordagem de modelagem por células,
assunto da seção \autoref{sec:modelagemredecelulas}.

Portanto, ficam aqui definidas as seguintes premissas:

\begin{enumerate}
  \item O reator opera adiabaticamente.
  \item Gradientes radiais são desprezíveis.
  \item Dispersões axiais foram negligenciadas; portanto, assumiu-se o
  escoamento empistonado para ambas as fases líquida e gasosa.
  \item A fase gasosa está em excesso; dessa forma, negligenciou-se a
  resistência a transferência de massa na fase gasosa.
  \item Fator de molhamento, atividade catalítica e densidade do leito
  uniformes.
  \item A transferência de calor entre as fases e no interior das partículas de
  catalisador foram desprezadas.
  \item A região do \emph{quench} é assumida como sendo um tambor \emph{flash}, que
  atinge o equilíbrio instantaneamente.
  \item A corrente de saída do primeiro leito mistura-se instantaneamente com a
  corrente de \emph{quench}.
  \item O reator opera em regime de borbulhamento.
  \item Reações reversíveis e isomerizações foram negligenciadas.
  \item As reações ocorrem somente na interface líquido-sólido.
  \item A desativação catalítica foi desprezada.
\end{enumerate}

Um dos motivos de se manter praticamente inalteradas as premissas adotadas por
\citeonline{Rojas2014a} foi o de realiar comparações entre os dois estudos.

Negligenciar os gradientes e dispersões de calor e massa, tanto radialmente
quanto axialmente, é uma recomendação muito comum encontrada na literatura
\cite{Ancheyta2011, Ranade2011, Froment2011} para simulações cujo objetivo é
prever o comportamento de um TBR, sob o ponto de vista de liberação de calor e
conversão das reações. Da mesma forma é a desconsideração da transferência de
calor entre as fases.

Como já mencionado no \autoref{chap:revisaobibliografica},
\citeonline{Rojas2014a} utilizaram as reações publicadas por
\citeonline{Hanika1999}. Isso pode ter motivado os autores a negligenciar a
resistencia à transferência de massa na fase gasosa.

As premissas assumidas para a região de \emph{quench} são bastante razoáveis, já
que normalmente os projetistas de reatores TBR desejam a corrente oriunda do
leito superior se misture rapidamente com a corrente de \emph{quench},
homogeneizando a carga para o segundo leito. Uma mistura inadequada entre as
correntes de entrada na região de \emph{quench} pode levar a caminhos
preferenciais e regiões de estagnação no leito inferior \cite{Ancheyta2011}.

O regime de operação por borbulhamento foi adotado para que as
equações hidrodinâmicas fossem as mesmas utilizadas por \citeonline{Rojas2014a}.
Contudo, uma verificação da validade dessa premissa é apresentada no
\autoref{chap:resultados}.

\section{Modelagem por Rede de Células} \label{sec:modelagemredecelulas}

Para utilizar a abordagem de modelagem por rede de células, cada leito
catalítico foi subdivido em células em seu comprimento, i.e., discretizado em
$N_{Disc}$ segumentos de leito. Cada seguimento de leito fica identificado por
um índice ($z$). A figura \autoref{fig:celula} ilustra uma célula de leito de
reator. Do ponto de vista reacional, cada célula representará um reator CSTR
associado a um tambor de \emph{flash}. Nesse tambor de \emph{flash}, mantém-se a premissa
de que o equilíbrio termodinâmico é atingido instantaneamente.

 \begin{figure}[htb]
 \centering \includegraphics[scale=0.75]{images/Chap3/celula.png}
 \caption{Elemento discretizado de leito - célula}
 \label{fig:celula}
 \end{figure}

\nomenclature{$N_{Disc}$}{Número de células de um leito do reator}
\nomenclature[S]{$z$}{z-ésima célula de leito de reator}

Para a região de quench, será usado um tambor de \emph{flash}, onde o equilíbrio
termodinâmico é também atingido instantaneamente. A \autoref{fig:quench} ilustra
a região de quench.

 \begin{figure}[htb]
 \centering \includegraphics[scale=0.75]{images/Chap3/quench.png}
 \caption{Região de quench do reator}
 \label{fig:quench}
 \end{figure}

Com esse tipo de abordagem, foi possível considerar os efeitos termodinâmicos
ao longo do reator, como também prever a composição das fases líquida e
gasosa. As próximas seções apresentam o equacionamento construído

\section{Balanço de Massa e Energia} \label{sec:balancomassaenergia}

Normalmente, para a modelagem de TBR heterogênea, os autores que utilizam
modelos determinísticos contínuos fazem o balanço de massa para todas as
fases separadamente, i.e., para cada fase é feito o balanço de massa por
componente. %\citeonline{Ancheyta2011} apresenta de uma forma didática as
% equações de balanço de massa de cada fase, considerando e explicando todos os
% efeitos (dispersões, reações, transferências de massa).
Cada pesquisador, portanto, de acordo com a premissa adotada, elimina muitos dos
termos das equações de balanço, a fim simplificar o modelo e facilitar a
solução. 

Para este trabalho, diante das premissas adotadas, a equação de balanço de massa
está mostrada na \autoref{eq:balancodemassa}. Nessa equação, foi considerado o
termo de diferenciação da massa $M$ de cada componente $i$, em cada célula $z$,
no tempo $t$. Dessa forma, é possível o estudo de respostas dinâmicas do
processo.

\begin{equation}
\dfrac{dM_{i,z}}{dt} = F_zC_{i,z} - F_{z+1}C_{i,z+1} +
\displaystyle\sum_{j=1}^{N_{reac}} \nu_{i,j}r_{j,z} \dfrac{W}{N_{Disc}}
\label{eq:balancodemassa}
\end{equation}

onde $M_{i,z}$ é a massa do componente $i$ na célula $z$, $j$ é o número
identificador de cada reação, $N_{reac}$ é o número total de reações,
$\nu_{i,j}$ é o coeficiente do componente $i$ na reação $j$, $r_{j,z}$ é taxa
da reação $j$ na célula $z$, $F$ é a vazão molar total e $W$ é a massa total de
catalisador presente no leito.

\nomenclature{$M$}{Massa \nomunit{kmol}}
\nomenclature{$N_{reac}$}{Número total de reações}
\nomenclature{$r$}{Taxa de reação \nomunit{kmol/(kg_{cat}.h)}}
\nomenclature{$F$}{Vazão molar \nomunit{kmol/h}}
\nomenclature{$W$}{Massa total de catalisador em um leito\nomunit{kg}}
\nomenclature[S]{$j$}{j-ésima reação}
\nomenclature[G]{$\nu$}{Coeficiente de reação}

A \autoref{eq:balancodeenergia} mostra o balanço de energia de forma
discretizada.

\begin{equation}
\dfrac{dE_{z}}{dt} = F_zh_{z} - F_{z+1}h_{z+1} +
\displaystyle\sum_{j=1}^{N_{reac}} \Delta H_{j}r_{j,z} \dfrac{W}{N_{Disc}}
\label{eq:balancodeenergia}
\end{equation}

onde $E_{z}$ é a energia na célula $z$,  $h$ é a entalpia da
corrente $F$ e $\Delta H_{j}$ o calor envolvido na reação $j$.

\nomenclature{$E$}{Energia \nomunit{kJ}}
\nomenclature{$h$}{Entalpia \nomunit{kJ/kmol}}
\nomenclature[G]{$\Delta H$}{Calor de reação \nomunit{kJ/kmol}}

As \autoref{eq:balancodemassa} e \autoref{eq:balancodeenergia}, somadas a
corrente $F$ e seu cálculo de ELV, permitem determinar a fração vaporizada,
composição das fases e respectivas propriedades físicas em cada célula $z$.

É importante notar ainda que, da forma como as \autoref{eq:balancodemassa} e
\autoref{eq:balancodeenergia} estão escritas, massa e energia contidas em cada
célula são grandezas extensivas. 

\section{Cinética das Reações} \label{sec:cineticadasreacoes}

As reações de hidrogenação e os parâmetros cinéticos estão na
\autoref{tab:composicao}. Conforme já explicado, o sistema reacional foi
considerado como sendo um conjunto de hidrogenações irreversíveis e
independentes da concentração de hidrogênio. A constante de pseudo-primeira
ordem da reação $j$ na célula $z$, $k^{'}_{j,z}$, está definido na
\autoref{eq:constantepseudoprimeiraordem}.

\begin{equation}
k^{'}_{j,z} = \eta_ik_jC^{*}_{H_2}
\label{eq:constantepseudoprimeiraordem}
\end{equation}

sendo $\eta$ o fator de efetividade de intradifusão, $k$ a
constante de taxa de reação instrínseca e $C^{*}_{H_2}$ a concentração de
equilíbrio de hidrogênio em fase líquida.

\nomenclature{$k^{'}$}{Constante de taxa de reação de pseudo-primeira
ordem \nomunit{kg/(kg_{cat}.h)}}
\nomenclature{$k$}{Constante de taxa de reação intrínseca
\nomunit{kg.m^3/(kg_{cat}.h.kmol_{H_2})}}
\nomenclature{$C^{*}_{H_2}$}{Concentração de equilíbrio de hidrogênio em fase
líquida \nomunit{kmol/m^3} \nomunit{kJ/kmol}}
\nomenclature[G]{$\eta$}{Fator de efetividade de intradifusão}

Portanto, a taxa $r$ da reação $j$ na célula $z$ será dada pela
\autoref{eq:taxareacao}:

\begin{equation}
r_{j,z} = k^{*}_{j,z}C^{S}_{i,z}
\label{eq:taxareacao}
\end{equation}

onde o vetor $C^{S}$ representa a concentração das espécies químicas na fase
sólida. Essa concentração é determinada pela transferência de massa
líquido-sólido, como está apresentado na \autoref{sec:interfaceliquidosolido}.

A constante da taxa de reação específica $k^{*}$ é calculada pela
equação de van't Hoff, na $T^{ref}$ = $417 K$, como segue:

\begin{equation}
k^{*}_{j,z} = \dfrac{k^{'}_{j,z}} {\rho^{L}_{z+1}} \exp
\left[{\dfrac{-Ea_j}{R} \left (\dfrac{1}{T_{z+1}} -
\dfrac{1}{T^{ref}} \right )}\right]
\label{eq:constantetaxareacaoespecifica}
\end{equation}

sendo $\rho^L$ a massa específica da fase líquida.

Nota-se na equação \autoref{eq:constantetaxareacaoespecifica} que foram usadas
temperatura e massa específica da corrente de saída da célula ($z+1$). Essa
opção foi feita de forma abritrária, já que poderiam ser usadas as condições de
entrada da mesma forma.

\nomenclature{$k^{*}$}{constante da taxa de reação específica
\nomunit{kmol/(m^3/(kg_{cat}.h)}}
\nomenclature[R]{$L$}{Designação da fase líquida}
\nomenclature[R]{$S$}{Designação da fase sólida}
\nomenclature{$T^{ref}$}{Temperatura de referência, $417 K$}
\nomenclature[G]{$\rho$}{Massa específica \nomunit{kg/m^3}}
\nomenclature{$T$}{Temperatura \nomunit{K}}
\nomenclature{$R$}{Constante universal dos gases ideais \nomunit{kJ/(kmol.K)}}
\nomenclature{$Ea$}{Energia de ativação \nomunit{kJ/kmol}}
\nomenclature[S]{$B$}{Leito catalítico}
\nomenclature{$C$}{Concentração \nomunit{kmol/m^3}}

\section{Termodinâmica} \label{sec:termodinamica}

A abordagem termodinâmica utilizada para prever o ELV foi a $\phi_i$ -
$\phi_i$, a mesma utilizada por \citeonline{Rojas2014a}, com a EoS de SRK
\cite{Soave1972}, e seguindo as recomendações e os parâmetros do trabalho feito
por \citeonline{Zhou2006} para a solubilidade de hidrogênio em gasolina de
pirólise. Esta recomendação consiste em utilizar a regra de mistura clássica de
van der Waals \cite{VanderWaals1873}, com parâmetros de interação binária
específicos para cada um dos pares de hidrogênio/hidrocarboneto. Para o caso da
regra de mistura escolhida, os parâmetros cruzados $aij$ estão definidos pela
\autoref{eq:parametroaij} \cite{Peng1976,Soave1972}.

\begin{equation}
a_{i,j} = \sqrt{a_ia_j}(1-\delta_{ij})
\label{eq:parametroaij}
\end{equation}

Para a avaliação do ELV e demais propriedades, foi utilizado o pacote
termodinâmico do simulador de processos iiSE (\emph{Industrial Integrated
Simulation Environment}). Para tanto, foi criada uma simulação no iiSE, onde
foram colocadas as composições da carga e da corrente de quench. O \emso, por
sua vez, utiliza a simulação criada em iiSE para os cálculos termodinâmicos, e
somente para este fim. 

Para finalizar esta seção, vale esclarecer que, da forma como foi implementada
no \emso, a corrente $F$ não só contém a informação de vazão e composição molar
(total e por componente), mas tabém possui uma rotina de cálculo de
\emph{flash}, utilizando como variáveis de entrada pressão, temperatura e
composição global. Dessa forma, ficam incluídos os efeitos de dissolução de
hidrogênio na fase líquida bem como a vaporização da fase líquida.

\nomenclature[G]{$\phi$}{Coeficiente de fugacidade}
\nomenclature[G]{$\delta_{ij}$}{Parâmetro de interação binária}
\nomenclature[Z]{iiSE}{\emph{Industrial Integrated Simulation Environment}}
\nomenclature{$a_{ij}$}{Parâmetro cruzado entre as espécies $i$ e $j$}
\nomenclature{$a$}{Parâmetro de atração}

\section{Interface Líquido-Sólido} \label{sec:interfaceliquidosolido}

Como a transferência de massa gás-líquido foi desprezada, compete aqui
apresentar as equações para determinar a concentração das espécies químicas
na superfície das partículas de catalisador, $C^S$. 

A primeira dessas equações é a \autoref{eq:transferenciamassa}, na qual a
transferência de massa na interface líquido-sólido de um reagente é igual a
velocidade com que ele é consumido reacionalmente:

\begin{equation}
k_{i,z}^{LS}a^{LS}(C^L_{i,z}-C^S_{i,z}) = \rho_B
\displaystyle\sum_{j=1}^{N_{reac}}
\nu_{i,j}r_{j,z}
\label{eq:transferenciamassa}
\end{equation}

sendo $C^{L}$ a concentração de qualquer espécie química presenta na fase
líquida e $k^{LS}$ o coeficiente de transferência de massa na interface líquido
sólido, calculado para cada reagente.

\nomenclature{$k^{LS}$}{Coeficiente de transferência de massa líquido-sólido
\nomunit{m/h}}

O parâmetro $a^{LS}$ é a área específica do catalisador é função do diâmetro das
partículas $d_p$ de catalisador e da porosidade do leito $\epsilon_B$:

\begin{equation}
a^{LS} = 6 \dfrac{(1-\epsilon_B)}{d_p}
\label{eq:aLS}
\end{equation}

\nomenclature{$a^{LS}$}{Área suferficial específica do catalisador
\nomunit{m^2/m^3}}

Para determinar $k_{LS}$, foi utilizada a equação de \citeonline{Hirose1976}
para o número de Sherwood dos componentes em fase líquida ($Sh^L$).

\begin{equation}
\epsilon_BSh^L_{i,z} = 0,8(Re^L_z)^{0,5}(Sc^L_{i,z})^{1/3} \qquad (Re<200)
\label{eq:Sh1}
\end{equation}

\begin{equation}
\epsilon_BSh^L_{i,z} = 0,53(Re^L_z)^{0,58}(Sc^L_{i,z})^{1/3} \qquad (Re>200)
\label{eq:Sh2}
\end{equation}

\begin{equation}
k_{LS,z} = \dfrac{Sh^L_{i,z}D^L_{i}}{d_p}
\label{eq:kLS}
\end{equation}

\begin{equation} 
Sc^L_{i,z} = \dfrac{\mu^L_{z}}{\rho^L_{z+1}D^L_{i,z}}
\label{eq:Sc}
\end{equation}

onde $Re^L$ é o número de Reynolds da fase líquida, $Sc$ é o número de Schmidt,
$D^L_{i}$ é difusividade molecular do composto $i$ na fase líquida e $\mu^L$ é
viscosidade da fase líquida.

\nomenclature{$Sh$}{Número de Sherwood}
\nomenclature{$Re$}{Número de Reynolds}
\nomenclature{$Sh$}{Número de Schmidt}
\nomenclature{$D^L$}{Difusividade molecular \nomunit{m^2/s}}
\nomenclature[G]{$\mu$}{Viscosidade \nomunit{cP}}

\section{Hidrodinâmica} \label{sec:hidrodinamica3}

Essa seção tem por objetivo apresentar as principais equações utilizadas no
cálculo de parâmetros hidrodinâmicos.

Para o cálculo da perda de carga, \citeonline{Rojas2014a} utilizaram uma equação
do tipo Ergun \apud{Ergun1952}{Holub1993}, com correções e constantes propostas
por \citeonline{Benkrid1997}. Essa equação é válida para regimes de alta
interação (borbulhamento, conforme a premissa adotada). Para o presente
trabalho, portanto, adotou-se a mesma equação, que é mostrada a seguir na forma
discretizada:

\begin{equation}
\Delta P
\label{eq:deltaP}
\end{equation}

onde $u$ é a velocidade da fase e $L_z$ é dado por $L/N_{Disc}$ de cada leito.

\nomenclature{u}{Velocidade superficial da fase \nomunit{m/s}}
\nomenclature[S]{R}{Reator}

Visto que a pressão de operação do reator é muito acima da pressão atmosféricas,
a equação usada para estimar a retenção de líquido $\epsilon^L$ foi a proposta
por \citeonline{Larachi1991}, que, \citeonline{Ranade2011}, é uma dentre outras
equações levantadas para sistemas de alta pressão \cite{Ancheyta2011}.

\begin{equation}
log \left (1-\dfrac{\epsilon_{z}^L}{\epsilon_B} \right) =
-\dfrac{1,22(We_{z}^L)^{0,15}}{(Re_{z}^L)^{0,20}(X_{z}^G)^{0,15}}
\label{eq:epsilonL}
\end{equation}

onde $We$ é o número de Webber e $X^G$ é o número de Lockhart-Martinelli
para a fase gás. Ambos são definidos a seguir.

\begin{equation}
We_{z}^L = \dfrac{(u_{z}^L)^2d_p\rho_{z}^L}{\sigma_{z}^L}
\label{eq:webber}
\end{equation}

\begin{equation}
X_{z}^G = \dfrac{u_{z}^G}{u_z^L} \sqrt{\dfrac{\rho^L}{\rho^G}}
\label{eq:X}
\end{equation}

sendo que $\sigma^L$ é a tensão superficial da fase líquida.

\nomenclature[G]{$\sigma$}{Tensão superficial \nomunit{N/m}}
\nomenclature{We}{Número de Webber}
\nomenclature{X}{Número de Lockhart-Martinelli}

Para verificar a premissa de que o molhamento do catalisador é completo
($\eta_{CE} = 1$), será utilizada a equação proposta por
\citeonline{Al-Dahhan1995} para sistemas de alta pressão, como segue:

\begin{equation}
\eta_{CE,z} = 1,104(Re_z^L)^{1/3} \left [
\dfrac{1 + [(\Delta P_z/L_{z})/(\rho_{z}^L g)]}{Ga_{z}^{L}} \right ]
\label{eq:molhamento}
\end{equation}

sendo $Ga$ o número de Galileo e $g$ a aceleração da gravidade.

\nomenclature{Ga}{Número de Galileo}
\nomenclature{g}{Aceleração da gravidade \nomunit{m/s^2}}

Um parâmetro utilizado na literatura para verificar se, na modelagem do reator,
o fenômeno de dispersão axial é relevante, é o número de Peclet ($Pe$).
A \autoref{eq:numerodepeclet} apresenta a correlação para o número de
Peclet utilizada neste trabalho, e que foi proposta por
\citeonline{Cassanello1992}.

\begin{equation}
Pe_z^L = \dfrac{L_z}{d_p} 2,3(Re_z^L)^{0,33}(Ga_{z}^{L})^{-0,19}
\label{eq:numerodepeclet}
\end{equation}

\nomenclature{Pe}{Número de Peclet}

\section{Implementação da Modelagem} \label{sec:implementacao}

Para a implementação da modelagem, foram usados dois softwares: iiSE e \emso. 

\subsection{iiSE} \label{sec:iise}

O iiSE é uma ferramenta de simulação desenvolvida pela empresa VRTech para a
simulação de processos químicos e petroquímicos. Além de ser possibilitar a
montagem das simulações de maneira gráfica, permite a comunicação com o
Microsoft Excel. Além disso, há também a possibilidade de exportar os resultados
de alguns equipamentos no formato .mso (para o simulador \emso).

Como já explicado na \autoref{sec:termodinamica}, o iiSe serviu como uma
ferramenta para o cálculo de ELV das correntes $F$, dados $T_z$, $P_z$ e
composição. Nele não foram implementadas quaisquer equações de balanço ou
correlações; ele apenas é chamado pelo \emso para solucionar os cálculos de
\emph{flash} necessários.

Ao pacote termodinâmico do iiSE foram adicionadas as constantes da equação de
SRK levantadas por \citeonline{Zhou2006} para a solubilidade de hidrogênio em
gasolina de pirólise.

\subsection{EMSO} \label{sec:EMSO}

O simulador \emso é uma ferramenta para modelagem, simulação e otimização de
sistemas, com foco principal em respostas dinâmicas. O \emso realiza a
verificação da consistência das unidades de medida, da solvabilidade do sistema
de equações e das condições iniciais. As três entidades principais dessa
linguagem de modelagem são: modelos (\emph{models}), equipamentos
(\emph{devices}) e fluxograma (\emph{flowsheet}) \cite{Soares2003}.

\textit{Modelos} são descrições matemáticas de um tipo de equipamento (bomba,
reator, corrente de fluxo); um \textit{equipamento} é uma instância de um de um
\textit{modelo}; e o \textit{fluxograma} representa o processo a ser analisado,
que é composto por um conjunto de \textit{equipamentos} \cite{Soares2003}.

O \emso já possui uma biblioteca de modelos prontos para o uso (EML -
\emph{\emso  Model Library}. Para a solução do presente trabalho, dois modelos
extras foram criados.

O primeiro modelo representa as correntes de fluxo que entram e saem de
cada célula, incluindo o cálculo de \emph{flash}. Esse modelo foi chamado de
\emph{streamTP.mso} e baseou-se nos modelos já criados para corrente na EML, com
a peculiaridade de receber como parâmetros de entrada temperatura, pressão e
composição. O código o objeto \emph{streamTP.mso} está no
\autoref{chap:streamTP}

O segundo modelo (\autoref{chap:modeloleitofixo}) foi o de um reator de leito
fixo. Nele é possível utilizar tanto fluxo monofásico quanto fluxo bifásico.
Esse modelo contém a declaração das principais variáveis, além das principais
equações (balanços de massa e energia, por exemplo).

Por fim, para unir todos os modelos e inserir equações e variáveis
específicas para a finalidade do presente trabalho, foi criado o fluxograma do
processo, que está disponível no \autoref{chap:fluxogramaprocesso}.

\nomenclature[Z]{EML}{\emso  Model Library}





































  
%
% 
%
\chapter{Resultados e Discussões} \label{chap:resultados}
% 

%
% 
%
\chapter{Conclusões e Sugestões} \label{chap:conclusoes}
% 
 
 
% Comandos para as referencias bibliográficas
\newdimen\bibindent
\setlength\bibindent{1.5em} % identaçao das referencias
{ % Um lineskip menor neste contexto de referencias
\baselineskip 4.3mm
% Edite o arquivo ppgeq.bib com as suas próprias referências 
\bibliography{ppgeq}
} 

% Inclusao dos Appendices 
\appendix
%
%
%
\chapter{Modelo escrito para correntes $F$ com cálculo de
\textit{flash}}
\label{chap:streamTP}

\lstinputlisting[firstline=1, lastline=55, numbers=none, language=EMSO,
label=code:streamTP] {images/Ap1/streamTP.mso}


%
%
%
\chapter{Modelo de reator de leito fixo}
\label{chap:modeloleitofixo}

\lstinputlisting[firstline=1, lastline=258, numbers=none, language=EMSO,
label=code:modeloleitofixo] {images/Ap1/FBR_bifasic_model_Pygas.mso}


%
%
%
\chapter{Código principal para solução do reator em estado estacionário}
\label{chap:codigoestacionario}

\lstinputlisting[firstline=1, lastline=664, numbers=none, language=EMSO,
label=code:codigoestacionario] {images/Ap1/Reactor.mso}


\end{document}
