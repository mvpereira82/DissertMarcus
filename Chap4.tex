%
% 
%
\chapter{Resultados e Discussões} \label{chap:resultados}
% 
% Como se sabe, a operação de colunas de destilação vem sendo estudada há muitos anos.
% Sabe-se também, que este
% equipamento é um dos mais importantes das indústrias químicas, petroquímicas e de alimentos. Sua
% importância não se resume só a sua função e operação, mas também devido ao fato de que
% colunas de destilação são responsáveis por grande parte do gasto de energia em uma
% indústria. Sendo assim, o desenvolvimento de modelos para estes equipamentos é uma tarefa
% relevante, tanto para projeto quanto para simulações de operação e otimização das
% unidades existentes.

Modelos matemáticos de colunas de destilação podem ser classificados de acordo com o seu
grau de detalhamento: se possuem predição da composição, temperatura e vazões para cada prato,
os chamados modelos rigorosos, ou se são compostos por uma descrição global da coluna
utilizando um menor número de variáveis, baseados em algum tipo de interpolação, os modelos
reduzidos. Neste trabalho, foram apresentadas as diversas formas de
modelagem de sistemas de separação e concluiu-se que a escolha das considerações
simplificativas varia com a aplicação do modelo. Em estudos onde são necessárias simulações
repetitivas, como em otimizações, os modelos reduzidos podem ser os indicados.
Mas, em estudos de partidas e paradas de colunas de destilação é importante que o
comportamento dinâmico do sistema seja bem representado levando ao uso de
modelos rigorosos completos. 

Na linha da modelagem baseada no equilíbrio termodinâmico entre as fases líquida e vapor, foram
desenvolvidos todos os modelos necessários para a confecção de um modelo genérico de coluna
de destilação. Os modelos foram implementados no simulador dinâmico de processos EMSO
utilizando seu ambiente de modelagem e sua linguagem própria. Os modelos gerados neste estudo
fazem parte da biblioteca EML (EMSO Model Library). Esta biblioteca é distribuída no conceito
de \emph{software} livre, disponibilizando todos os modelos via internet e sem
custo.

Um exemplo de uma coluna de destilação reativa serviu como ilustração para a aplicação
dos modelos desenvolvidos neste trabalho em partidas de unidades de separação.
Ao contrário da modelagem encontrada na literatura para esse tipo de sistema, a condição de
equilíbrio termodinâmico entre as fases líquida e vapor foi considerada desde o início da
simulação sem causar problemas de inicialização e integração, conforme alguns autores relatam.
A partida de uma
coluna vazia e fria foi simulada de acordo com três das estratégias convencionais encontradas
na literatura. Como esperado, a partida realizada com refluxo total para a coluna apresentou um
tempo de transientes maior que as demais. Além disso, o modelo apresentado se mostrou muito
indicado para simulações de partida, uma vez que possíveis dificuldades para a determinação de
seus principais parâmetros não afetam a análise do tempo de operação dinâmica.

Outro exemplo foi utilizado para a validação do modelo desenvolvido. Trata-se de uma coluna
deisobutanizadora da PETROBRAS composta de 80 pratos que separa isobutano de
uma mistura de 13 componentes. Algumas pequenas adaptações foram realizadas nos modelos e os resultados obtidos
foram satisfatórios. Tanto a resposta estacionária quanto o comportamento dinâmico da
unidade foram comparados. O problema gerou um sistema de mais de 6300 equações. Um período de
8 dias de operação com várias perturbações na carga e no refluxo da coluna foi simulado em
cerca de 19 minutos de CPU. Conclui-se com isto que este modelo pode ser aplicado para os
mais diversos fins, desde simulações de procedimentos de parada e partidas de unidades, até
estimações, otimizações e treinamento de operação.

Como perspectivas de trabalhos futuros, algumas melhorias podem ser feitas nos modelos apresentados.
Para tornar o modelo de prato ainda mais genérico, pode ser adicionada uma corrente material extra
correspondente a uma retirada lateral. Isto permitiria a modelagem de colunas de destilação de petróleo
sem utilizar separadores de correntes entre os estágios da coluna.
A modelagem fiel do controle de pressão do topo da torre através do \textit{hot bypass} também é um assunto
interessante a ser estudado. Provavelmente esta implementação melhoraria a capacidade de predição do modelo
principalmente nas condições do topo da coluna.
Além disso, alguns assuntos importantes sobre a modelagem
de sistemas de separação podem ser estudados.
Sabe-se que na indústria, grande parte das colunas de destilação operam muito próximo
do seu limite superior de capacidade. Assim, a previsão do fenômeno de inundação da torre
e suas conseqüências na eficiência de separação se torna muito interessante.
Além de colunas de destilação de pratos, as colunas recheadas também são muito utilizadas.
A adaptação dos modelos apresentados nesta dissertação para a
representação deste tipo de colunas deverá ser a continuidade deste trabalho.

