%
% 
%
\chapter{Resultados e Discussão} \label{chap:resultados}

\section{Avaliações Preliminares} \label{sec:avaliacoesespreliminares}

\subsection{Consistência dos Dados de Literatura}
\label{sec:dadosliteratura}

Após a implementação da metodologia descrita no \autoref{chap:moddesenvolvidos},
os resultados encontrados foram significativamente diferentes daqueles apresentados
por \citeonline{Rojas2014a}. A \autoref{fig:perfilTvalidacao}, por
exemplo, compara o perfil de temperatura dos leitos reais (de planta) com os
resultados da simulação implementada por \citeonline{Rojas2014a} e com os
resultados do trabalho aqui apresentado.

\begin{figure}[htb]
\centering \includegraphics[scale=0.4]{images/Chap4/perfilTvalidacao.png}
\caption{Perfil de temperatura dos leitos.}
\label{fig:perfilTvalidacao}
\end{figure}

Algumas questões precisam ser destacadas da \autoref{fig:perfilTvalidacao}.
A primeira delas refere-se ao resfriamento na região zona de \emph{quench}. O
dado de planta apresenta um resfriamento de \SI{25,7}{K} entre os leitos
catalíticos; a simulação de \citeonline{Rojas2014a} apresenta resfriamento de
\SI{21,6}{K} para a região; e o valor encontrado no
presente trabalho foi de \SI{4,1}{K}. Essa situação torna-se mais enigmática ao
se constatar que o balanço de energia proposto por \citeonline{Rojas2014a} para
a região de \emph{quench} é ainda mais simplificado que o proposto neste estudo.

Uma análise do valor de LHSV (\emph{Liquid Hourly Space Velocity}) foi feita
na tentativa de trazer um esclarecimento às discrepâncias encontradas. O
parâmetro LHSV pode ser definido como: 
\begin{equation}
\textrm{LHSV} = F_{vol}^L/{V_{R}}
\label{eq:LHSV}
\end{equation}
onde $F_{v}^L$ é a vazão volumétrica em \si{m^3/h} de líquido e $V_{R}$ é o
volume total do reator (somados os dois leitos).

\nomenclature{$F_v$}{Vazão volumétrica \nomunit{m^3/h}}
\nomenclature[S]{$v$}{Referente ao volume}
\nomenclature{$V$}{Volume \nomunit{m^3}}
\nomenclature{$V_R$}{Volume total do reator \nomunit{m^3}}

A \autoref{tab:comparacaoLHSV} mostra alguns valores encontrados na literatura
para LHSV, incluindo o trabalho de \citeonline{Rojas2014a}. Os valores de vazão
de líquido nesta tabela referem-se à carga combinada e reciclo, se houver.

\begin{table}[!htb]
\begin{center}
\caption{Comparação entre valores de LHSV calculados com base em dados
publicados na literatura.}
\label{tab:comparacaoLHSV}
\small
\begin{tabular}{lccc}
{Autor} & {$F_v^L$ (\si{m^3/h})} & {$V_R$ (\si{m^3})} &
{LHSV (\si{1/h})}
\\
\hline
{\citeonline{Arpornwichanop2008}} & \num{69} & \num{60} & \num{1,14} \\
{\citeonline{Mederos2007}} & \num{165} & \num{62} & \num{2,66} \\
{\citeonline{Rojas2014a}} & \num{347} & \num{48} & \num{7,28} \\
\bottomrule
\end{tabular}
\end{center}
\end{table}

Comparativamente fica claro que o valor de LHSV do trabalho de
\citeonline{Rojas2014a} está uma ordem de grandeza acima dos valores
comumente empregados nos projetos de reatores TBR. Infelizmente, as tentativas
de contato com os autores do trabalho de \citeonline{Rojas2014a} não lograram
exito até o momento. Sendo assim, não foi possível realizar a verificação dos
valores por eles publicados.

Sendo assim, foi necessário ajustar o valor da vazão de alimentação de
gasolina de pirólise para que os dados apresentassem coerência. O critério
utilizado foi o grau de resfriamento que ocorre na região de \emph{quench}. De
maneira mais direta, a vazão da carga combinada (hidrogênio + \emph{pygas} + diluente) foi
modificada para que a diferença de temperatura entre a saída do primeiro leito e
entrada do segundo fossem semelhantes aos dados de planta e, também, aos demais
dados calculados por \citeonline{Rojas2014a}. Em outras palavras, a vazão de
carga do reator foi modificada (a menor) para ficar compatível com o valor da
corrente de \emph{quench}, adequando o grau de resfriamento na região de
\emph{quench} do reator.

Após várias tentativas, o valor de $F_{w,1}$ foi alterado para \SI{2,079e4}
{kg/h}. Com essa nova vazão, tanto o resfriamento da região de \emph{quench}
quando a exotermia dos leitos tornaram-se aderentes aos dados de planta e de
literatura, como será mostrado na \autoref{sec:estadoestacionario}.

A \autoref{tab:comparacaoLHSV2} repete os valores da
\autoref{tab:comparacaoLHSV}, mas agora com a inclusão do valor de LHSV
após o ajuste da vazão de carga do reator.

\begin{table}[!htb]
\begin{center}
\caption{Comparação entre LHSVs publicadas.}
\label{tab:comparacaoLHSV2}
\small
\begin{tabular}{lccc}
{Autor} & {$F_v^L$ (\si{m^3/h})} & {$V_R$ (\si{m^3})} &
{LHSV (\si{1/h})}
\\
\hline
{\citeonline{Arpornwichanop2008}} & \num{69} & \num{60} & \num{1,14} \\
{\citeonline{Mederos2007}} & \num{165} & \num{62} & \num{2,66} \\
{\citeonline{Rojas2014a}} & \num{347} & \num{18} & \num{7,28} \\
{Dados ajustados} & \num{76} & \num{48} & \num{1,59} \\
\bottomrule
\end{tabular}
\end{center}
\end{table}

\subsection{Determinação do Valor de $N_{Disc}$} \label{sec:determinacaoNDisc}

Nesta seção será apresentada a maneira como foi escolhido o valor ideal de
$N_{Disc}$. O valor escolhido foi utilizado em todas as simulações doravante
apresentadas.

A maneira mais simples (e talvez a mais adequada) de se determinar o número de
células é simular o reator com diferentes valores de $N_{Disc}$ e, então,
verificar os resultados de algumas variáveis chave. Duas variáveis
foram escolhidas: temperatura e fração molar global de hidrogênio.

A \autoref{fig:NDiscT} mostra o perfil de temperatura do leito superior, para
diferentes valores de $N_{Disc}$. Percebe-se que, entre $N_{Disc} = 4$ e
$N_{Disc} = 8$ ainda há alguma diferença no comportamento do perfil de
temperatura. Contudo, é imperceptível a diferença entre os perfis resultantes de
$N_{Disc} = 8$ e $N_{Disc} = 16$. 

\begin{figure}[htb] \centering
\includegraphics[scale=0.4]{images/Chap4/NDiscT.png}
\caption{Avaliação de NDisc pelo perfil de temperatura.}
\label{fig:NDiscT}
\end{figure}

Observando o perfil da fração molar global de hidrogênio de diferentes valores
de $N_{Disc}$ (também para o leito superior), é ainda menos relevante a
diferença entre $N_{Disc} = 4$ e $N_{Disc} = 8$, como mostra a
\autoref{fig:NDiscz}.

\begin{figure}[htb]
\centering \includegraphics[scale=0.4]{images/Chap4/NDiscz.png}
\caption{Avaliação de NDisc pela fração molar total de hidrogênio.}
\label{fig:NDiscz}
\end{figure}

Assim, discretizar cada leito catalítico em \num{8} células pareceria uma
escolha razoável e suficiente para o objetivo deste trabalho. Entretanto, para
garantir a representação dos leitos como contínuos, foi utilizado um valor um
pouco maior, $N_{Disc}$ = \num{10}. Esta escolha faz com que o modelo apresente
um total de \num{9299} variáveis.

Mesmo não considerando a perda de carga nos resultados finais, as equações
necessárias para os cálculos dela foram mantidas. Além disso, algumas das
equações hidrodinâmicas influenciam em parâmetros de transferência de massa, que
influenciam na taxa da reação (Cs). Outras referem-se a avaliação do padrão de
escoamento. Dentro das \num{9299}, portanto, estão todas as equações, mesmo as
menos relevantes para os resultados.

\subsection{Avaliação da Perda de Carga} \label{sec:avaliacaoperdadecarga}

A equação da perda de carga quando implementada no modelo do reator causou
alguns problemas de convergência. Assim, optou-se primeiramente por avaliar a
importância da perda de carga, e se ela alteraria muito o perfil de pressão a
ponto de impactar no cálculo das propriedades das fases e do ELV.

Foi feita, portanto, uma simulação inicial que estimou a perda de carga, mas sem
alterar o perfil de pressão (pressão constante ao longo do reator). A perda de
carga estimada para os leitos está na \autoref{tab:perdadecarga}. Por esses
valores, percebe-se que a pressão ao longo dos leitos permanece praticamente
inalterada. Sendo assim, o reator foi simulado desprezando-se a perda de carga.

\begin{table}[!htb]
\begin{center}
\caption{Avaliação da perda de carga dos leitos.}
\label{tab:perdadecarga}
\small
\begin{tabular}{lccc}
{} & {$\Delta P/L$ (\si{atm/m})} & {Pressão de entrada (\si{atm})} & {Pressão de saída
(\si{atm})}
\\
\hline
{Leito 1} & \num{3,4e-3} & \num{49,64} & \num{49,63} \\
{Leito 2} & \num{3,7e-3} & \num{49,63} & \num{49,61} \\
\bottomrule
\end{tabular}
\end{center}
\end{table}

Um valor tão baixo assim de perda de carga não é o comumente encontrado na
indústria. Porém, a equação utilizada por \citeonline{Rojas2014a} e aqui neste
trabalho pode não ser a mais adequada, já que houve uma alteração da vazão de
carga do reator. Está mostrado na \autoref{sec:estadoestacionario} que o regime
de escoamento do reator se altera de regime de bolha (alta interação),
considerado por \citeonline{Rojas2014a}, para regime de escoamento gotejante
(baixa interação). Assim sendo, a equação para avaliação da perda de carga no
reator deveria ter sido outra mais adequada às características do sistema.

\section{Estado Estacionário} \label{sec:estadoestacionario}

\subsection{Perfil de Temperatura} \label{sec:perfildetemperatura}

A \autoref{fig:perfilT} mostra o perfil de temperatura dos leitos
calculado pela modelagem aqui proposta, além daquele calculado por
\citeonline{Rojas2014a} e pelos dados de planta por eles publicados.  

\begin{figure}[htb]
\centering \includegraphics[width=0.8\textwidth]{images/Chap4/perfilT.png}
\caption{Perfil de temperatura dos leitos após ajuste na vazão de carga.}
\label{fig:perfilT}
\end{figure}

É possível perceber pela \autoref{fig:perfilT} que, mesmo após o ajuste da vazão
de carga, a modelagem aqui desenvolvida apresenta exotermia nos leitos inferior
aos dados de planta. Observando os valores de entalpias de reação usados por
\citeonline{Rojas2014a}, eles apresentam-se inferiores aos valores publicados
por \citeonline{Skala2003} (hidrogenação do ciclopentadieno) e
\citeonline{Nijhuis2003} (hidrogenação de gasolina de pirólise).

Sendo assim, foi feito um ajuste de $10 \%$ no valor das entalpias de reação.
Dessa forma, o perfil de temperatura torna-se semelhante aos dados de planta
publicados por \citeonline{Rojas2014a}. A \autoref{fig:perfilT2} repete os
valores de temperatura ao longo do reator (dados calculados e de planta, e
apresenta o perfil de tempeartura calculado aqui após os ajustes citados.

\begin{figure}[htb]
\centering \includegraphics[width=0.8\textwidth]{images/Chap4/perfilT2.png}
\caption{Perfil de temperatura dos leitos após ajustes.}
\label{fig:perfilT2}
\end{figure}

Sendo o perfil de temperatura obtido neste trabalho semelhante tanto aos dados
de planta quando aos valores calculados por \citeonline{Rojas2014a}, é possível
apresentar os demais resultados e efetuar mais comparações.

\subsection{Composição da Corrente Efluente do Reator}
\label{composicaodacorrenteefluentedoreator}

\citeonline{Rojas2014a} divulgaram a composição do efluente do reator agrupando
alguns compostos. Assim sendo, foi implementado o mesmo agrupamento para fins de
comparação, como mostra a \autoref{tab:composicaodoefluente}. Pela comparação
entre os dados composicionais do efluente do reator, corrobora-se a opção de se
alterar a vazão de alimentação.

\begin{table}[!htb]
\begin{center}
\caption{Composição do Efluente do Reator em Fração Molar.}
\label{tab:composicaodoefluente}
\small
\begin{tabular}{lccc}
{Compostos} & {Dados de Planta} & {\citeonline{Rojas2014a}} & {Este Trabalho}
\\
\hline
{Hidrogênio} & \num{0,06} & \num{0,05} & \num{0,04} \\
{C1-C4 não reativos} & \num{0,05} & \num{0,04} & \num{0,04} \\
{C6-C9 não reativos} & \num{0,06} & \num{0,07} & \num{0,06} \\
{Benzeno-Tolueno-Xileno} & \num{0,48} & \num{0,47} & \num{0,48} \\
{Dienos C5-C6} & \num{0,00} & \num{0,00} & \num{0,00} \\
{Olefinas C5-C6} & \num{0,14} & \num{0,15} & \num{0,14} \\
{Parafinas C5-C6} & \num{0,10} & \num{0,12} & \num{0,14} \\
{Estirenos} & \num{0,00} & \num{0,00} & \num{0,00} \\
{Aromáticos reativos} & \num{0,06} & \num{0,07} & \num{0,07} \\
{Diciclodienos} & \num{0,00} & \num{0,00} & \num{0,00} \\
{Diciclodienos hidrogenados} & \num{0,03} & \num{0,03} & \num{0,03} \\
\bottomrule
\end{tabular}
\end{center}
\end{table}

\subsection{ELV} \label{elv}

Como está mencionado no \autoref{chap:moddesenvolvidos}, a fração de vapor
em cada célula $n$ é determinada por um cálculo de \emph{flash}, implementado no
modelo \code{streamTP}, detalhado no \autoref{chap:streamTP}. A
\autoref{fig:perfilfracaovaporizada} mostra o perfil da fração de vapor (base
molar) ao longo do reator.

\begin{figure}[htb]
\centering \includegraphics[scale=0.4]{images/Chap4/perfilfracaovaporizada.png}
\caption{Perfil de fração vaporizada.}
\label{fig:perfilfracaovaporizada}
\end{figure}

É possível perceber pela \autoref{fig:perfilfracaovaporizada} que a fração
vaporizada na entrada do primeiro leito calculada neste trabalho é muito próxima
daquela calculada por \citeonline{Rojas2014a}. Já para o segundo leito, há uma
diferença próxima a 20\% entre os dois conjuntos de resultados.

Duas questões entre o presente trabalho e o publicado por
\citeonline{Rojas2014a} podem justificar essa diferença. A primeira delas é a
questão da perda de carga. Neste trabalho, como explicado na
\autoref{sec:avaliacaoperdadecarga}, a perda de carga fora desprezada. Visto que
\citeonline{Rojas2014a} não publicaram o valor da perda de carga encontrada nos
leitos catalíticos, não é possível avaliar se há ou não uma influência da perda
de carga na fração vaporizada. Há ainda outra diferença: o parâmetro de
interação binaria $\delta_{ij}$ foi calculada por \citeonline{Rojas2014a}
utilizando a base de dados do software comercial PRO/II 9.0. Ainda, a regra de
mistura por eles apontada como a de melhor resultado (e, entende-se como sendo a
que foi utilizada para gerar os resultados publicados) foi a de PA-RESimSci
(SimSci-Esscor. PRO/II 9.0 Component and Thermophysical Properties Reference
Manual (2010)).

\subsection{Comportamento do Hidrogênio} \label{comportamentodohidrogenio}

Uma das premissas adotadas aqui, e que é oriunda do trabalho de
\citeonline{Rojas2014a}, é que a concentração de hidrogênio na
fase líquida é aproximadamente constante ao longo do reator. Por essa razão,
outra premissa também foi considerada, a de que as reações podem ser
consideradas como de pseudo-primeira ordem, desprezando a concentração de
hidrogênio.

A \autoref{fig:perfilfracaoh2liquido} mostra que a fração molar de hidrogênio em
fase líquida aumenta ao longo do reator. Esse aumento é explicado pelo
aumento da temperatura ao longo dos leitos, provocado pela liberação de calor
das reações de hidrogenação. A \autoref{fig:perfilfracaoh2temperatura} apresenta
os mesmos resultados de fração molar de hidrogênio na fase líquida apresentados
na \autoref{fig:perfilfracaoh2liquido}, mas desta vez referente à temperatura de
cada célula dos leitos do reator (relembrando que cada célula \emph{n} tem uma
temperatura e um comprimento atrelado a ela).

\begin{figure}[htb]
\centering \includegraphics[scale=0.4]{images/Chap4/perfilfracaoh2liquido.png}
\caption{Perfil de fração molar de hidrogênio em fase líquida.}
\label{fig:perfilfracaoh2liquido}
\end{figure}

\begin{figure}[htb]
\centering
\includegraphics[scale=0.4]{images/Chap4/perfilfracaoh2temperatura.png}
\caption{Fração molar de hidrogênio em fase líquida \emph{vs.} da temperatura.}
\label{fig:perfilfracaoh2temperatura}
\end{figure}

A \autoref{fig:perfilfracaoh2gas} mostra o comportamento da
fração molar de hidrogênio na fase gasosa. Por ser consumido conforme as reações
avançam ao longo dos leitos catalíticos, o hidrogênio tende a diminuir de
concentração na fase gasosa ao longo do reator, como já era esperado.

\begin{figure}[htb]
\centering
\includegraphics[scale=0.4]{images/Chap4/perfilfracaoh2gas.png}
\caption{Perfil de fração de hidrogênio em fase gasosa.}
\label{fig:perfilfracaoh2gas}
\end{figure}

\subsection{Propriedades Termodinâmicas} \label{propriedadestermodinâmicas}

A \autoref{tab:propriedadestermodinamicas} apresenta os valores de algumas
propriedades termodinâmicas calculadas neste trabalho.

\begin{table}[!htb]
\begin{center}
\caption{Propriedades Termodinâmicas.}
\label{tab:propriedadestermodinamicas}
\small
\begin{tabular}{lcccc}
{Propriedade} & {Entrada Leito 1} & {Saída Leito 1} & {Entrada Leito 2} &
{Saída Leito 2}
\\
\hline
{$MW^{L}$ (\si{kg/kmol})} & \num{82,4} & \num{81,5} & \num{81,7} & \num{81,1} \\
{$MW^{G}$ (\si{kg/kmol})} & \num{7,0} & \num{20,8} & \num{13,6} & \num{21,7} \\
{$\rho^{L}$(\si{kg/m^3})} & \num{728,7} & \num{621,3} & \num{659,1} &
\num{619,0}
\\
{$\rho^{G}$ (\si{kg/m^3})} & \num{11,4} & \num{29,4} & \num{22,0} & \num{31,0}
\\
\bottomrule
\end{tabular}
\end{center}
\end{table}

Conforme já esperado, o resultado da simulação mostra que a densidade do
líquido diminui ao longo do primeiro leito catalítico (conforme há aumento de
temperatura), aumenta na região de \emph{quench} (devido à injeção de líquido
frio), e volta a diminuir ao longo do segundo leito. A densidade da fase gasosa
tem o comportamento inverso, já que mesmo com o aumento da temperatura, a
densidade da fase gasosa aumenta a medida que sua massa molar aumenta.

Mudança mais relevante sofre a massa molar da fase gasosa, que aumenta ao
longo do primeiro e segundo leitos, diminuindo na zona de \emph{quench}. Esse
comportamento é reflexo tanto da diminuição da concentração de hidrogênio na
fase gasosa, que tende a diluir-se mais facilmente em fase líquida conforme a
temperatura aumenta, quanto da evaporação de compostos leves conforme o fluxo
aquece ao longo do leito catalítico.

A massa molar do líquido tende a permanecer constante ao longo de todo o
reator, mostrando que o aumento da concentração de hidrogênio não é relevante
para alterar significativamente essa propriedade.

\subsection{Propriedades Hidrodinâmicas} \label{propriedadeshidrodinâmicas}

A \autoref{tab:propriedadeshidrodinamicas} mostra o resultado de algumas
propriedades hidrodinâmicas e números adimensionais utilizados nos cálculos de
transferência de massa e perda de carga.

\begin{table}[!htb]
\begin{center}
\caption{Propriedades Hidrodinâmicas.}
\label{tab:propriedadeshidrodinamicas}
\small
\begin{tabular}{lcccc}
{Propriedade} & {Entrada Leito 1} & {Saída Leito 1} & {Entrada Leito 2} &
{Saída Leito 2}
\\
\hline
{$u^{L}$ (\si{m/s})} & \num{3,2e-3} & \num{3,6e-3} & \num{4,7e-3} & \num{4,9e-3}
\\
{$u^{G}$ (\si{m/s})} & \num{3,3e-3} & \num{1,7e-3} & \num{1,5E-3} & \num{0,5E-3} \\
{$\mu^{L}$(\si{cP})} & \num{0,128} & \num{0,103} & \num{0,130} & \num{0,121} \\
{$\mu^{G}$ (\si{cP})} & \num{0,012} & \num{0,013} & \num{0,012} & \num{0,013} \\
{$Re^{L}$} & \num{54,4} & \num{67,6} & \num{72,1} & \num{78,0} \\
{$Re^{G}$} & \num{10,6} & \num{10,1} & \num{7,6} & \num{3,0} \\
{$Ga^{L}$} & \num{8,6e6} & \num{10,4e6} & \num{7,0e6} & \num{7,4e6} \\
{$We^{L}$} & \num{1,5e-3} & \num{2,8e-3} & \num{4,1e-3} & \num{5,2e-3} \\
{$X^{G}$} & \num{7,8} & \num{2,3} & \num{1,8} & \num{0,5} \\
{$Pe^{L}$} & \num{428} & \num{443} & \num{600} & \num{609} \\
\bottomrule
\end{tabular}
\end{center}
\end{table}

É preciso informar aqui que os valores de $\mu^L$ e $\mu^G$ apresentados na
\autoref{tab:propriedadeshidrodinamicas} foram calculadas por rotinas internas
do simulador iiSE; os demais parâmetros foram calculados por equações
apresentadas na \autoref{chap:moddesenvolvidos}. Ademais, o comportamento da
viscosidade das fases foi como o esperado, ou seja, a viscosidade da fase
líquida diminui com o aumento da temperatura, e a viscosidade da fase gasosa
aumenta com o mesmo aumento (considerando que a pressão ao longo do reator é
constante).

Um resultado que chama a atenção na \autoref{tab:propriedadeshidrodinamicas}
refere-se à velocidade de escoamento das fases e, consequentemente, ao número de
Reynolds delas. A velocidade da fase líquida divulgada por
\citeonline{Rojas2014a}, por exemplo, esteve entre \SI{1,2e-2}{m/s} e
\SI{1,5e-2}{m/s}, o que é cerca de cinco vezes maior do que os valores
encontrados neste trabalho. Aqui, portanto, fica clara a divergência da vazão de
carga do reator utilizada neste trabalho e no trabalho original.

Os números de $Ga^L$ e $We^L$ foram usados para o cálculo de $\eta_{CE}$ e
$\epsilon^{L}$, respectivamente. Esses parâmetros estão apresentados nas seções
a seguir.

\subsubsection{Retenção Total de Líquido $\epsilon^{L}$}
\label{retencaototaldeliquido}

A \autoref{fig:perfilretencaoliquido} mostra o perfil da retenção de líquido
total $\epsilon^{L}$ ao longo dos leitos. Considerando que, ao longo do leito, a
fração vaporizada diminui (\autoref{tab:comparacaoLHSV2}), é coerente que a
retenção de líquido aumente. Do primeiro leito para o segundo leito há um
aumento da retenção de líquido, que é fruto do aumento da vazão de líquido
dentro do reator causado pela injeção da corrente líquida de \emph{quench}.

\begin{figure}[htb]
\centering
\includegraphics[scale=0.4]{images/Chap4/perfilretencaoliquido.png}
\caption{Perfil de $\epsilon^{L}$.}
\label{fig:perfilretencaoliquido}
\end{figure}

Como explicado no \autoref{chap:moddesenvolvidos}, a variável $\epsilon^{L}$
possui como dimensão o valor de $N_{Disc}$.

Os valores $\epsilon^{L}$ encontrados por \citeonline{Rojas2014a} foram de
\num{0,14} e \num{0,17} para a entrada do Leito 1 e saída do Leito 2,
respectivamente. A diferença entre os resultados pode ser explicada pela
incerteza quanto à vazão de carga no reator, que neste trabalho precisou ser
modificada por motivos já explicados. Contudo, a diferença entre os trabalhos
não é tão significativa assim, sendo ambos aderentes aos resultados
encontrados em literatura \cite{Ancheyta2011}.

\subsubsection{Eficiência de Molhamento} \label{eficienciademolhamento}

A eficiência de molhamento $\eta_{CE}$ para reator aqui estudado foi estimada em
\SI{0,71}-\SI{0,76} para o Leito 1 e \SI{0,81}-\SI{0,83} para o Leito 2. Em seu
trabalho, \citeonline{Rojas2014a} encontraram valores maiores do que \SI{1,0}
para o parâmetro $\eta_{CE}$.

Este é mais um caso onde a incerteza quanto à vazão de carga do reator
impossibilita uma análise comparativa mais acurada. De qualquer forma, vale a
observação de que, por possuir uma vazão de líquido maior devido à injeção da
corrente de \emph{quench}, o segundo leito do reator apresenta eficiêcia de
molhamento maior do que o primeiro, como já era esperado.

\subsubsection{Escoamento - Classificação} \label{escoamentoclassificacao}

Segundo \citeonline{Sie1998}, o regime de escoamento gotejante é definido
principalmente pelas velocidades superficiais das fases gás e líquida. Neste
trabalho há uma figura em que, de acordo com as velocidades superficiais das
fases líquida e gasosa (em \si{m/s}), é possível verificar o tipo de escoamento
no qual o reator opera. Segundo essa figura, o reator estudado neste trabalho
enquadra-se como um reator de leito gotejante.

\citeonline{Saroha1996} publicaram um mapa mais elaborado que classifica o
escoamento dos reatores trifásicos de leito fixo. De acordo com esse mapa, o
reator objeto deste estudo é classificado com escoamento tipo leito
gotejante, como mostra a figura a seguir:

\begin{figure}[htb]
\centering
\includegraphics[scale=0.4]{images/Chap4/validacaoescoamento.png}
\caption{Verificação do perfil de escomento (extraído de
\citeonline{Rojas2014a})}
\label{fig:validacaoescoamento}
\end{figure}

Assim, o tipo de escomento encontrado aqui para o reator é contrário à premissa
adotada de escoamento por borbulhamento, sugerido por \citeonline{Rojas2014a}. É
preciso lembrar que este trabalho preocupou-se em manter ao máximo as premissas
e valores publicados por \citeonline{Rojas2014a}. Contudo, a premissa adotada
para o tipo de escoamento (borbulhamento), que influencia na escolha das
equações hidrodinâmicas, deveria ser revista e outras equações hidrodinâmicas
deveriam ter sido utilizadas no lugar das atuais.

\subsection{Transferência de Massa - Interface Líquido-Sólido}
\label{transferênciademassainterfaceliquidosolido}

Como assumido nas premissas deste trabalho, a transferência de massa entre as
fases gasosa e líquida foi desprezada, ficando então a composição \emph{bulk}
destas fases definida pelo ELV.

Já a composição dos reagentes e produtos na superfície do catalisador ($C^S$)
foi considerada distinta da composição \emph{bulk} da fase líquida $C^L$. A
\autoref{tab:concentraçõesparaoleito1} mostra as concentrações dos compostos na
fase líquida (\emph{bulk}) e na superfície do catalisador para a entrada e a
saída do Leito 1. De acordo com os resultados, fica evidente que a diferença dos
valores de $C^S$ e $C^L$ para cada um dos compostos é mínima.

Assim, a taxa da reação poderia muito bem ser calculada com a composição
\emph{bulk} da fase líquida $C^L$, desprezando-se a transferência de massa na
interface líquido-sólido.

\begin{table}[!htb]
\begin{center}
\caption{Concentrações $C^L_{i}$ e $C^S_{i}$ para o Leito 1.}
\label{tab:concentraçõesparaoleito1}
\small
\begin{tabular}{lcccc}
{Composto} & {$C^{L}_{1}$} & {$C^{S}_{1}$} & {$C^{L}_{11}$} & {$C^{S}_{11}$}
\\
\hline
{Hidrogênio} & \num{0,2397} & \num{0,2367} & \num{0,2656} & \num{0,2644} \\
{Metano} & \num{0,1002} & \num{0,1002} & \num{0,1304} & \num{0,1304} \\
{Etano} & \num{0,0210} & \num{0,0210} & \num{0,0208} & \num{0,0208} \\
{n-Propano} & \num{0,0511} & \num{0,0511} & \num{0,0468} & \num{0,0468} \\
{n-Butano} & \num{0,0352} & \num{0,0352} & \num{0,0312} & \num{0,0312} \\
{n-Pentano} & \num{0,5327} & \num{0,5328} & \num{0,5070} & \num{0,5073} \\
{trans-2-Penteno} & \num{0,5512} & \num{0,5519} & \num{0,5876} & \num{0,5879} \\
{trans-1,3-Pentadieno} & \num{0,2455} & \num{0,2447} & \num{0,0739} &
\num{0,0734}
\\
{Ciclopentano} & \num{0,1557} & \num{0,1559} & \num{0,1911} & \num{0,1914} \\
{Ciclopenteno} & \num{0,2539} & \num{0,2554} & \num{0,3054} & \num{0,3053} \\
{Metil-1,3-Ciclopentadieno} & \num{0,1611} & \num{0,0210} & \num{0,0208} &
\num{0,0208}
\\
{n-Hexano} & \num{0,2757} & \num{0,2757} & \num{0,2424} & \num{0,2424} \\
{Metilciclopentano} & \num{0,1383} & \num{0,1383} & \num{0,1370} & \num{0,1371}
\\
{Metilciclopenteno} & \num{0,1866} & \num{0,1871} & \num{0,2428} & \num{0,2430}
\\
{1,3-Ciclopentadieno} & \num{0,1695} & \num{0,1680} & \num{0,0128} &
\num{0,0126}
\\
{Benzeno} & \num{2,6803} & \num{2,6803} & \num{2,3579} & \num{2,3579} \\
{n-Heptano} & \num{0,2029} & \num{0,2029} & \num{0,1786} & \num{0,1786} \\
{Tolueno} & \num{1,2642} & \num{1,2642} & \num{1,1141} & \num{1,1141} \\
{n-Octano} & \num{0,0825} & \num{0,0825} & \num{0,0727} & \num{0,0727} \\
{Etilbenzeno} & \num{0,2536} & \num{0,2546} & \num{0,3146} & \num{0,3148} \\
{Estireno} & \num{0,1207} & \num{0,1197} & \num{0,0156} & \num{0,0155} \\
{Xileno} & \num{0,3712} & \num{0,3712} & \num{0,3277} & \num{0,3277} \\
{n-Nonano} & \num{0,0376} & \num{0,0376} & \num{0,0332} & \num{0,0332} \\
{1-Metil-3-Etilbenzeno} & \num{0,1813} & \num{0,1816} & \num{0,2081} &
\num{0,2083}
\\
{Metilestireno} & \num{0,0813} & \num{0,0810} & \num{0,0239} & \num{0,0237} \\
{Dihidrodiciclopentadieno} & \num{0,2003} & \num{0,2014} & \num{0,2721} &
\num{0,2724} \\
{Diciclopentadieno} & \num{0,1262} & \num{0,1251} & \num{0,0164} & \num{0,0161}
\\
\bottomrule
\end{tabular}
\end{center}
\end{table}

\section{Respostas Dinâmicas} \label{sec:respostasdinamicas}

O modelo desenvolvido neste trabalho é próprio para estudos e
avaliações de respostas dinâmicas. Três avaliações de respostas dinâmicas foram
feitas, a saber:

\begin{enumerate}
  \item Resposta a   um degrau na temperatura da carga, de \SI{366}{K} para
  \SI{300}{K} - Avaliação da importância da inércia térmica do catalisador;
  \item Resposta a um degrau na vazão da corrente de quench, de \SI{20790}{kg/h}
  para \SI{10790}{kg/h}; e
  \item Resposta a um degrau na vazão da corrente de quench, de \SI{20790}{kg/h}
  para \SI{30790}{kg/h}.
\end{enumerate}

As seções a seguir apresentam uma breve discussão para cada um dos casos
mencionados.

\subsection{Resposta a   um degrau na temperatura da carga, de \si{366}{K} para
\si{300}{K} - Avaliação da importância da inércia térmica do catalisador}
\label{sec:respostaaumdegrautemp}

Este primeiro caso reflete o comportamento do reator no caso da perda de
aquecimento da carga, seja por falta de utilidades (vapor para um trocador de
calor, por exemplo), seja por falha nos elementos primários de uma eventual
malha de controle. Como consequência dessa perda de aquecimento da carga, a
exotermia vista em operação normal diminui drasticamente, sinal de que a
conversão dos reagentes passou a patamares inaceitáveis. O perfil de temperatura
dos leitos do reator após o degrau de temperatura na carga está mostrado na
\autoref{fig:Feed_T_300K}.

A \autoref{fig:Feed_T_300K} apresenta resultados que não consideram a influência
do catalisador na inércia térmica de cada célula, i.e., considera a
\autoref{eq:holdupenergia} para o cálculo da energia acumulada em cada célula.
Dessa forma, o tempo até o reator atingir o novo estado estacionário após o
degrau foi de aproximadamente \SI{900}{s} ou \SI{15}{min}. Ao ser calculada a
energia acumulada em cada célula pela \autoref{eq:holdupenergiadinamica}, que
considera a presença do catalisador, o tempo para se atingir o novo estado
estacionário passou a ser de aproximadamente \SI{3000}{s} ou \SI{50}{min}.
\autoref{fig:Feed_T_300K_catalyst_influence} mostra o perfil de temperatura dos
leitos do reator após o degrau de temperatura na carga, considerando a inércia
térmica do catalisador.
\begin{figure}[htb]
\centering
\includegraphics[scale=0.8]{images/Chap4/Feed_T_300K.png}
\caption{Resposta dinâmica do perfil de tempeartura a um degrau (a menor) de
temperatura na carga.}
\label{fig:Feed_T_300K}
\end{figure}
\begin{figure}[htb]
\centering
\includegraphics[scale=0.8]{images/Chap4/Feed_T_300K_catalyst_influence.png}
\caption{Resposta dinâmica do perfil de tempeartura a um degrau (a menor) de
temperatura na carga, considerando a inércia térmica do catalisador.}
\label{fig:Feed_T_300K_catalyst_influence}
\end{figure}
\citeonline{Arpornwichanop2002} simularam um reator de hidrogenação de
gasolina de pirólise onde, no balanço de energia, a inércia térmica dos leitos
catalíticos nas respostas dinâmicas foi desprezada. Segundo os autores, o tempo
necessário para se atingir o estado estacionário a partir das condições
iniciais foi de \SI{700}{s}. Esse valor está em consonância com o que foi
simulado aqui, quando desconsiderada a massa de catalisador na inércia térmica
do sistema (\SI{900}{s}). \citeonline{Arpornwichanop2002} não publicaram dados
de planta que corroborassem a modelagem dinâmica proposta. Contudo, é possível
concluir que a massa de catalisador influencia fortemente no tempo de resposta
de reatores de hidrogenação de \emph{pygas}.
% A concentração de diolefinas na saída do reator passa de \si{0,0305}{kmol/m^3}
% para \si{0,2868}{kmol/m^3}; 
% 
% \begin{figure}[htb]
% \centering
% \includegraphics[scale=0.8]{images/Chap4/Feed_T_300K_catalyst_influence_dienes.png}
% \caption{Resposta dinâmica da concentração de diolefinas a um degrau (a menor)
% de temperatura na carga.}
% \label{fig:Feed_T_300K_catalyst_influence_dienes}
% \end{figure}
A diminuição da temperatura de entrada faz com que a fração molar
total de estireno na saída do reator aumente de \num{1,96e-4} para \num{0,0026},
cerca de \num{13,3} vezes maior. Considerando que o estireno é um composto que
precisa ser eliminado da nafta de pirólise, a queda de temperatura em questão é
inaceitável. A \autoref{fig:Feed_T_300K_catalyst_influence_styrene} mostra como
seria a resposta dinâmica do perfil da concentração de estireno ao degrau de
temperatura (a menor) da carga.
\begin{figure}[htb]
\centering
\includegraphics[scale=0.8]{images/Chap4/Feed_T_300K_catalyst_influence_styrene.png}
\caption{Resposta dinâmica da concentração de estireno a um degrau (a menor) de
temperatura na carga.}
\label{fig:Feed_T_300K_catalyst_influence_styrene}
\end{figure}

A concentração de diolefinas na saída do reator atinge o estado estacionário em
valor superior ao estado estacionário anterior. Porém, a tendência de diminuição
dessa concentração se mantém entre os estados estacionários. O comportamento da
concentração de diolefinas está mostrado na
\autoref{fig:Feed_T_300K_catalyst_influence_diolefins}.
 
\begin{figure}[htb]
\centering
\includegraphics[scale=0.8]{images/Chap4/Feed_T_300K_catalyst_influence_diolefins.png}
\caption{Resposta dinâmica da concentração de diolefinas a um degrau (a menor)
de temperatura na carga.}
\label{fig:Feed_T_300K_catalyst_influence_diolefins}
\end{figure}

Diferentemente da concentração de diolefinas, a tendência da concentração de
olefinas se altera no leito 2 do reator, passando a aumentar ao longo do leito.
Isso se deve ao fato de haver diminuição da conversão de olefinas em parafinas
ante a taxa de hidrogenação de diolefinas. A resposta do perfil de concentração
de olefinas ao degrau de temperatura estudado aqui está na
\autoref{fig:Feed_T_300K_catalyst_influence_olefins}.

\begin{figure}[htb]
\centering
\includegraphics[scale=0.8]{images/Chap4/Feed_T_300K_catalyst_influence_olefins.png}
\caption{Resposta dinâmica da concentração de olefinas a um degrau (a menor) de
temperatura na carga.}
\label{fig:Feed_T_300K_catalyst_influence_olefins}
\end{figure}

As duas próximas avaliações de respostas dinâmicas foram feitas considerando o
catalisador no cálculo da energia acumulada nas células.

\subsection{Resposta a um degrau na vazão da corrente de \emph{quench}, de
\SI{20790}{kg/h} para \SI{10790}{kg/h}}
\label{sec:respostaaumdegrauvazao1}

A perda de parte da vazão da corrente de \emph{quench} tem efeito imediato no
Leito 2 do reator, como mostra a \autoref{fig:Quench_F_10790_catalyst_influence}.

\begin{figure}[htb]
\centering
\includegraphics[scale=0.8]{images/Chap4/Quench_F_10790_catalyst_influence.png}
\caption{Resposta dinâmica a um degrau (a menor) na vazão de \emph{quench}.}
\label{fig:Quench_F_10790_catalyst_influence}
\end{figure}

É importante notar aqui que a perda parcial da vazão de \emph{quench} poderia
levar algumas regiões do reator a temperaturas mais altas do que o
necessário para o objetivo do processo. Ainda, a exotermia
observada significa também maior conversão de olefinas que, para o objetivo
desse processo (produzir gasolina automotiva), tem efeito deletério quanto à
octanagem.

\subsection{Resposta a um degrau na vazão da corrente de \emph{quench}, de
\SI{20790}{kg/h} para \SI{30790}{kg/h}} \label{sec:respostaaumdegrauvazao3}

Para a situação em que a vazão da corrente de \emph{quench} aumenta, quer seja
por problema de instrumentação ou equipamento, quer seja por erro humano, o
Leito 2 é esfriado como mostra a Figura \autoref{fig:Quench_F_30790_catalyst_influence}.

\begin{figure}[htb]
\centering
\includegraphics[scale=0.8]{images/Chap4/Quench_F_30790_catalyst_influence.png}
\caption{Resposta dinâmica a um degrau (a maior) na vazão de \emph{quench}.}
\label{fig:Quench_F_30790_catalyst_influence}
\end{figure}

Um problema aqui, para esse caso, é a queda na conversão dos
compostos diolefínicos. Considerando que o reator em questão é de hidrogenação
parcial (voltado para produção de gasolina automotiva), a concentração de
compostos diolefínicos conferiria uma característica instável, indesejável para
a formulação de gasolina automotiva.
