%
\chapter{Modelo Desenvolvido}
\label{chap:moddesenvolvidos}

\section{Descrição do Processo} \label{sec:descricaoprocesso}

Como já mencionado no \autoref{chap:introducao}, o processo modelado
neste trabalho utilizou os dados publicados por \citeonline{Rojas2014a} tais
como: composição das correntes de carga, \emph{quench} e produto; dados
construtivos do reator; e dados de processo.

\subsection{Diagrama Esquemático} \label{sec:diagramaesquematico}

O reator é composto de dois leitos catalíticos, carregados de forma
densa (\emph{dense loading}). No topo de ambos os leitos há catalisador carregado de
forma solta (\emph{sock loading}). A carga é composta de gasolina de pirólise,
hidrogênio e parte do produto hidrogenado reciclado (diluente). Este último é
parte da corrente líquida oriunda de um vaso de separação, cuja carga é a corrente de
saída do reator. Parte da corrente líquida do vaso de separação também funciona
como corrente de resfriamento, que é injetada entre os dois leitos catalíticos
para controle de temperatura (\emph{quench}). Um diagrama simplificado que
ilustra o reator está na \autoref{fig:esquemareator}. A
\autoref{tab:dadosreator} apresenta as dimensões do reator.

\begin{figure}[htb]
\centering \includegraphics[scale=0.75]{images/Chap3/esquemareatorb.png}
\caption{Diagrama esquemático do reator \cite{Rojas2014a}.}
\label{fig:esquemareator}
\end{figure}

\begin{table}[!htb]
\begin{center}
\caption{Dados do Reator \cite{Rojas2014a}.}
\label{tab:dadosreator}
\small
\begin{tabular}{lcc}
{Dimensão} & {Variável} & {Valor}
\\
\hline
{Diâmetro do Reator} & {$D_R$} & \SI{3,047}{m} \\
{Comprimento do Leito Solto 1} & {$L_{sl1}$} & \SI{0,13}{m} \\
{Comprimento do Leito Denso 1} & {$L_{dl1}$} & \SI{2,97}{m} \\
{Comprimento do Leito Solto 2} & {$L_{sl2}$} & \SI{0,38}{m} \\
{Comprimento Leito Denso 2} & {$L_{dl2}$} & \SI{2,97}{m} \\
{Zona de Quench} & {$L_{Q}$} & \SI{1,55}{m} \\
\bottomrule
\end{tabular}
\end{center}
\end{table}

\nomenclature{$L_{sl1}$}{Comprimento do leito solto 1 \nomunit{m}}
\nomenclature{$L_{dl1}$}{Comprimento do leito denso 1 \nomunit{m}}
\nomenclature{$L_{sl2}$}{Comprimento do leito solto 2 \nomunit{m}}
\nomenclature{$L_{dl2}$}{Comprimento do leito denso 2 \nomunit{m}}
\nomenclature{$L_{Q}$}{Comprimento da zona de quench \nomunit{m}}

\subsection{Catalisador} \label{sec:catalisador}

O catalisador considerado na modelagem contém Pd suportado em alumina
(\ce{Al2O3}), de formato esférico. Além desses dados, foi fornecida a informação
de volume total de poros $Vt_{poros}$ = \SI{5,5e-4}{\cubic\meter\per\kg}. O
dado de volume total de poros não é suficiente por si só para calcular a
porosidade dos leitos, como é apresentado na \autoref{sec:propriedadesleitoscataliticos}.
Assim, adotou-se a massa específica do \ce{Al2O3} puro ($\rho_{\ce{Al2O3}}$ =
\SI{3940} {kg/m^3}) para a massa específica da parte sólida das partículas de
catalisador.

\nomenclature{$Vt_{poros}$}{Volume total de poros \nomunit{m^3/kg}}
\nomenclature[G]{$\rho_{\ce{Al_2O_3}}$}{Massa específica do óxido de alumínio
\nomunit{kg/m^3}}

\subsection{Porosidade dos Leitos Catalíticos}
\label{sec:propriedadesleitoscataliticos}

Antes de determinar a porosidade dos leitos catalíticos, é preciso avaliar a
massa específica das partículas ($\rho_{p}$). Este parâmetro é estimado em
\SI{1244}{kg/m^3} pela seguinte equação:
\begin{equation}
\rho_{p} = \dfrac{1}{Vt_{poros}+\dfrac{1}{\rho_{\ce{Al2O3}}}}
\label{eq:rhoparticula}
\end{equation}

\nomenclature[G]{$\rho_{p}$}{Massa específica da partícula de catalisador
\nomunit{kg/m^3}}

A porosidade de cada um dos leitos pode ser calculada como segue:
\begin{equation}
\epsilon_{B} = {1 - \dfrac{\rho_{B}}{\rho_{\ce{Al2O3}}}}
\label{eq:porosidade}
\end{equation}
sendo $V_l$ o volume do leito e $\rho_B$ a massa específica de um leito
catalítico logo após o seu carregamento.

Utilizando os dados apresentados na \autoref{sec:catalisador}, e utilizando os
valores de massa específica \emph{bulk} dos leitos $\rho_{B}$ publicados por
\citeonline{Rojas2014a}, chega-se nos valores mostrados na
\autoref{tab:dadosdosleitos}.

\begin{table}[!htb]
\begin{center}
\caption{Dados dos Leitos Catalíticos \cite{Rojas2014a}.}
\label{tab:dadosdosleitos}
\small
\begin{tabular}{ccc}
{ - } & {Densidade \emph{bulk} (\si{kg/m^3})} & {Porosidade} 
\\
\hline
{Leito 1} & \num{814} & \num{0,346} \\
{Leito 2} & \num{813} & \num{0,347} \\
\bottomrule
\end{tabular}
\end{center}
\end{table}

Visto que o topo dos leitos, carregados de maneira solta, não são partes
significativas dos leitos, a porosidade destas seções foi
considerada a mesma de carregamento denso para fins de simplificação. 

A massa de catalisador contida em cada leito catalítico pode ser expressa,
então, por:
\begin{equation}
W = \rho_pA_RL_{dl}(1-\epsilon_{B})
\label{eq:massacatalisador}
\end{equation}
sendo $A_R$ a área da seção transversal do reator.

\nomenclature{$A_R$}{Área da seção transversal do reator \nomunit{m^2}}

Para o Leito 1, a massa foi estimada em \SI{18389}{kg}; para o Leito 2, o valor
é de \SI{22601}{kg}.

\subsection{Composição das Correntes} \label{sec:composicaocorrentes}

A composição das correntes de entrada e de \emph{quench} estão na
\autoref{tab:composicao}. Para o presente trabalho, utilizou-se
apenas os valores da primeira corrida publicados por \citeonline{Rojas2014a}, já
que para este caso foram publicados também dados industriais.

As propriedades termodinâmicas dos compostos $28$ e $29$ não foram
encontradas. Assim, eles foram considerados como sendo parte dos compostos
$26$ e $27$, respectivamente.

\begin{table}[!htb]
\begin{center}
\caption{Composição das correntes de entrada e de \emph{quench}
\cite{Rojas2014a}.}
\label{tab:composicao}
\small
\begin{tabular}{clcc}
{Identificador $i$} & {Composto} & Entrada (\% mássica) & Quench (\% mássica)
\\
\hline
1 & Hidrogênio				& \num{0,48} & \num{0,08} \\
2 & Metano					& \num{0,52} & \num{0,70} \\
3 & Etano					& \num{0,12} & \num{0,11} \\
4 & n-Propano				& \num{0,36} & \num{0,27} \\
5 & n-Butano				& \num{0,30} & \num{0,24} \\
6 & n-Pentano				& \num{5,40} & \num{5,60} \\
7 & trans-2-Penteno			& \num{5,30} & \num{7,60} \\
8 & trans-1,3-Pentadieno	& \num{2,50} & \num{0,23} \\
9 & Ciclopentano			& \num{1,50} & \num{2,60} \\
10& Ciclopenteno			& \num{2,10} & \num{3,00} \\
11& Metil-1,3-Ciclopentadieno	& \num{1,90} & \num{0,21} \\
12& n-Hexano				& \num{3,30} & \num{3,30} \\
13& Metilciclopentano		& \num{1,60} & \num{1,70} \\
14& Metilciclopenteno		& \num{2,00} & \num{2,60} \\
15& 1,3-Ciclopentadieno		& \num{1,90} & \num{0,02} \\
16& Benzeno					& \num{28,90} & \num{30,10} \\
17& n-Heptano				& \num{2,80} & \num{2,90} \\
18& Tolueno					& \num{16,00} & \num{16,30} \\
19& n-Octano				& \num{1,30} & \num{1,30} \\
20& Etilbenzeno				& \num{3,40} & \num{5,40} \\
21& Estireno				& \num{2,00} & \num{0,11} \\
22& Xileno					& \num{5,40} & \num{5,50} \\
23& n-Nonano				& \num{0,66} & \num{0,72} \\
24& 1-Metil-3-Etilbenzeno	& \num{2,90} & \num{3,90} \\
25& Metilestireno			& \num{1,40} & \num{0,31} \\
26& Dihidrodiciclopentadieno	& \num{1,80} & \num{3,00} \\
27& Diciclopentadieno		& \num{1,80} & \num{0,12} \\
28& Metildihidrodiciclopentadieno	& \num{1,50} & \num{2,20} \\
29& Metildiciclopentadieno	& \num{0,88} & \num{0,11} \\
\bottomrule
\end{tabular}
\end{center}
Onde $i$ é o número identificador do composto na simulação a ser apresentada.
\end{table}

\nomenclature[S]{$i$}{i-ésimo componente}

Considerando a composição da corrente de carga apresentada na
\autoref{tab:composicao}, a relação entre a vazão de hidrogênio e da carga
(combinada com o reciclo) é de \SI{40,97}{Nm^3/m^3}

\subsection{Condições de operação} \label{sec:condicaocomposicaocorrentes}

As condições de operação consideradas inicialmente estão na
\autoref{tab:condicoesoperacao}. No \autoref{chap:resultados} há uma discussão
sobre a validade de algumas dessas variáveis aqui apresentadas.

\begin{table}[!htb]
\begin{center}
\caption{Dados do Reator \cite{Rojas2014a}.}
\label{tab:condicoesoperacao}
\small
\begin{tabular}{lcc}
{Dimensão} & {Variável Discretizada} & {Valor}
\\
\hline
{Pressão de entrada} & {$P_{1}$} & \SI{5,03}{MPa} \\
{Temperatura de entrada} & {$T_{1}$} & \SI{366,15}{K} \\
{Vazão mássica de entrada} & {$F_{w,1}$} & \SI{2,477e5}{kg/h} \\
{Pressão da corrente de \emph{quench}} & {$P_{Q}$} & \SI{4,6}{MPa} \\
{Temperatura da corrente de \emph{quench}} & {$T_{Q}$} & \SI{326,15}{K} \\
{Vazão mássica da corrente de \emph{quench}} & {$F_{w,Q}$} & \SI{2,079e4}{kg/h}
\\
\bottomrule
\end{tabular}
\end{center}
\end{table}

\nomenclature{$F_w$}{Vazão mássica \nomunit{kg/h}}
\nomenclature[S]{$w$}{Referente à massa}

\section{Hipóteses do Modelo} \label{sec:premissas}

Antes de apresentar as premissas que nortearam o presente trabalho, vale
aqui apresentar, para efeito de comparação, as premissas utilizadas por
\citeonline{Rojas2014a}.

\begin{enumerate}
  \item O reator opera em estado estacionário e adiabaticamente;
  \item Gradientes radiais são desprezíveis;
  \item Dispersões axiais foram negligenciadas; portanto, assumiu-se o
  escoamento empistonado para ambas as fases líquida e gasosa;
  \item A fase gasosa está em excesso; dessa forma, negligenciou-se a
  resistência à transferência de massa na fase gasosa;
  \item Fator de molhamento, atividade catalítica e densidade do leito
  uniformes;
  \item A transferência de calor entre as fases e no interior das partículas de
  catalisador foram desprezadas;
  \item A entalpia de dissolução de compostos na fase gás, bem como calor de
  vaporização de líquido foram desprezadas;
  \item A região de \emph{quench} é assumida como sendo um tambor \emph{flash},
  que atinge o equilíbrio instantaneamente;
  \item A corrente de saída do primeiro leito mistura-se instantaneamente com a
  corrente de \emph{quench};
  \item O reator opera em regime de borbulhamento;
  \item Reações reversíveis e isomerizações foram negligenciadas;
  \item As reações ocorrem somente na interface líquido-sólido;
  \item A desativação catalítica foi desprezada.
\end{enumerate}

Como já mencionado no \autoref{chap:introducao}, um dos objetivos do presente
trabalho é avaliar algumas respostas dinâmicas ao processo proposto. Além disso,
a implementação da modelagem foi feita em \emso, software que avalia respostas
dinâmicas por concepção. Assim, convenientemente, a premissa de um reator
operando em estado estacionário foi descartada. Permanece, porém, aqui, a
premissa de que o equipamento opere de forma adiabática.

A alteração mais importante às premissas adotadas por \citeonline{Rojas2014a}, e
que representa um avanço por eles publicado, é a consideração tanto da entalpia
de dissolução do hidrogênio na fase líquida quanto da entalpida de vaporização
de líquido. Isso foi possível graças à abordagem de modelagem por células,
assunto da \autoref{sec:modelagemredecelulas}.

Portanto, todas as hipóteses do trabalho de \citeonline{Rojas2014a} foram
mantidas, exceto a hipótese $7$. Um dos motivos de se manter praticamente
inalteradas as premissas adotadas por \citeonline{Rojas2014a} foi o de realizar
comparações entre os dois estudos.

% \begin{enumerate}
%   \item O reator opera adiabaticamente;
%   \item Gradientes radiais são desprezíveis;
%   \item Dispersões axiais foram negligenciadas; portanto, assumiu-se o
%   escoamento empistonado para ambas as fases líquida e gasosa;
%   \item A fase gasosa está em excesso; dessa forma, negligenciou-se a
%   resistência à transferência de massa na fase gasosa;
%   \item Fator de molhamento, atividade catalítica e densidade do leito
%   uniformes;
%   \item A transferência de calor entre as fases e no interior das partículas de
%   catalisador foram desprezadas;
%   \item A região de \emph{quench} é assumida como sendo um tambor \emph{flash},
%   que atinge o equilíbrio instantaneamente;
%   \item A corrente de saída do primeiro leito mistura-se instantaneamente com a
%   corrente de \emph{quench};
%   \item O reator opera em regime de borbulhamento;
%   \item Reações reversíveis e isomerizações foram negligenciadas;
%   \item As reações ocorrem somente na interface líquido-sólido;
%   \item A desativação catalítica foi desprezada.
% \end{enumerate}

Negligenciar os gradientes e dispersões de calor e massa, tanto radialmente
quanto axialmente, é uma recomendação muito comum encontrada na literatura
\cite{Ancheyta2011, Ranade2011, Froment2011} para simulações cujo objetivo é
prever o comportamento de um TBR, sob o ponto de vista de liberação de calor e
conversão das reações. Da mesma forma é a desconsideração da transferência de
calor entre as fases.

Como já mencionado no \autoref{chap:revisaobibliografica},
\citeonline{Rojas2014a} utilizaram as reações publicadas por
\citeonline{Hanika1999}. Isso pode ter motivado os autores a negligenciar a
resistência à transferência de massa na fase gasosa.

As premissas assumidas para a região de \emph{quench} são bastante razoáveis, já
que normalmente os projetistas de reatores TBR desejam que a corrente oriunda do
leito superior se misture rapidamente com a corrente de \emph{quench},
homogeneizando a carga para o segundo leito. Uma mistura inadequada entre as
correntes de entrada na região de \emph{quench} pode levar a caminhos
preferenciais e regiões de estagnação no leito inferior \cite{Ancheyta2011}.

O regime de operação por borbulhamento foi adotado para que as
equações hidrodinâmicas fossem as mesmas utilizadas por \citeonline{Rojas2014a}.
Contudo, uma verificação da validade dessa premissa é apresentada no
\autoref{chap:resultados}.

\section{Modelagem por Rede de Células} \label{sec:modelagemredecelulas}

Para utilizar a abordagem de modelagem por rede de células, cada leito
catalítico foi subdivido em células em seu comprimento, i.e., discretizado em
$N_{Disc}$ segmentos de leito. Cada segmento de leito fica identificado por
um índice ($n$). A \autoref{fig:celula} ilustra uma célula de leito de
reator. Do ponto de vista reacional, cada célula representará um reator CSTR
associado a um tambor de \emph{flash}. Nesse tambor de \emph{flash}, mantém-se a
premissa de que o equilíbrio termodinâmico é atingido instantaneamente.

 \begin{figure}[htb]
 \centering \includegraphics[scale=0.75]{images/Chap3/celula.png}
 \caption{Elemento discretizado de reator - célula}
 \label{fig:celula}
 \end{figure}

\nomenclature{$N_{Disc}$}{Número de células de um leito do reator}
\nomenclature[S]{$n$}{z-ésima célula de leito de reator}

É importante notar que as variáveis que fazem parte das correntes que
entram e saem das células na \autoref{fig:celula} (temperatura,
pressão, composição, fração vaporizada, fração dos compostos em fase líquida e
gasosa, por exemplo) são vetores de dimensão $N_{Disc}+1$.
Ao passo que as variáveis avaliadas internamente em cada célula (massa e energia
totais contidas em cada célula, taxa e entalpia des reações, parâmetros
hidrodinâmicos, por exemplo) possuem dimensão $N_{Disc}$.

Para a região de \emph{quench}, será usado um tambor de \emph{flash}, onde o
equilíbrio termodinâmico é também atingido instantaneamente. A
\autoref{fig:quench} ilustra a região de \emph{quench}.

 \begin{figure}[htb]
 \centering \includegraphics[scale=0.75]{images/Chap3/quench.png}
 \caption{Região de quench do reator.}
 \label{fig:quench}
 \end{figure}

Com esse tipo de abordagem, foi possível considerar os efeitos termodinâmicos
ao longo do reator, como também prever a composição das fases líquida e
gasosa. As próximas seções apresentam o equacionamento construído.

\section{Balanços de Massa e Energia} \label{sec:balancomassaenergia}

Normalmente, para a modelagem heterogênea de TBR, os autores que
utilizam modelos determinísticos contínuos fazem o balanço de massa para todas as
fases separadamente, i.e., para cada fase é feito o balanço de massa por
componente. Desta forma, elimina-se muitos dos termos das equações de balanço a
fim simplificar o modelo e facilitar a solução. 

Para este trabalho, diante das premissas adotadas, a equação de balanço de massa
considera o termo de diferenciação da massa $M$ de cada componente $i$, em cada
célula $n$, no tempo $t$: 
\begin{equation}
\dfrac{dM_{i,n}}{dt} = F_n z_{i,n} - F_{n+1} z_{i,n+1} +
\displaystyle\sum_{j=1}^{N_{reac}} \nu_{i,j}r_{j,n} \dfrac{W}{N_{Disc}}
\label{eq:balancodemassa}
\end{equation}
onde $M_{i,n}$ é a massa do componente $i$ na célula $n$, $j$ é o número
identificador de cada reação, $N_{reac}$ é o número total de reações,
$\nu_{i,j}$ é o coeficiente estequiométrico do componente $i$ na reação $j$,
$r_{j,n}$ é taxa da reação $j$ na célula $n$, $F$ é a vazão molar total,
$z$ é a fração molar global do composto $i$ e $W$ é a massa total de
catalisador presente no leito.

Com a \autoref{eq:balancodemassa} é possível o estudo de respostas dinâmicas do
processo estudado.

\nomenclature{$M$}{Massa \nomunit{kmol}}
\nomenclature{$N_{reac}$}{Número total de reações}
\nomenclature{$r$}{Taxa de reação \nomunit{kmol/(kg_{cat}.h)}}
\nomenclature{$F$}{Vazão molar \nomunit{kmol/h}}
\nomenclature{$W$}{Massa total de catalisador em um leito\nomunit{kg}}
\nomenclature[S]{$j$}{j-ésima reação}
\nomenclature[G]{$\nu$}{Coeficiente de reação}
\nomenclature{$z$}{Fração molar global de um composto}
\nomenclature{$y$}{Fração molar de um composto em fase gasosa}
\nomenclature{$x$}{Fração molar de um composto em fase líquida}

De forma similar, o seguinte balanço de energia de forma
discretizada foi considerado:
\begin{equation}
\dfrac{dE_{n}}{dt} = F_nh_{n} - F_{n+1}h_{n+1} +
\displaystyle\sum_{j=1}^{N_{reac}} \Delta H_{j}r_{j,n} \dfrac{W}{N_{Disc}}
\label{eq:balancodeenergia}
\end{equation}
onde $E_{n}$ representa a energia acumulada na célula $n$, $h$ é a entalpia da
corrente $F$ e $\Delta H_{j}$ o calor envolvido na reação $j$.

Para a expressão da energia acumulada em cada célula, a seguinte relação
foi considerada para o estado estacionário:
\begin{equation}
E_{n} = M_{n}h_{n+1} - P_{n+1}V_{n}
\label{eq:holdupenergia}
\end{equation}
onde $V_{n} = AL_{leito}\epsilon_{B}/N_{Disc}$ é o volume disponível para as
fases líquida e gasosa na célula $n$.

Para avaliação de respostas dinâmicas, a relação entre a energia aculumada em
cada célula é a seguinte:
\begin{equation}
E_{n} = M_{n}h_{n+1} - P_{n+1}V_{n} + 
\dfrac{W}{N_{Disc}}C_{p,p}T_{n}
\label{eq:holdupenergiadinamica}
\end{equation}
onde $C_{p,p}$ é a capacidade calorífica do catalisador, \SI{880}{J/(kg K)}.

A \autoref{eq:holdupenergia} mostra que, para obtenção dos resultados em estado
estacionário, apenas as fases líquida e gasosa interferem na inercia térmica do
sistema. Já a \autoref{eq:holdupenergiadinamica} apresenta a inserção do
catalisador na inércia térmica do sistema. A relevância dessa consideração é
discutida no \autoref{chap:resultados}.

\nomenclature{$E$}{Energia \nomunit{kJ}}
\nomenclature{$h$}{Entalpia \nomunit{kJ/kmol}}
\nomenclature{$C_{p,p}$}{Capacidade calorífica a pressão constante do
catalisador \nomunit{J/(kg.K)}}
%\nomenclature{$T^{ref}_2$}{Temperatura de referência, \SI{320,15}{K}}
\nomenclature{$h$}{Entalpia \nomunit{kJ/kmol}}
\nomenclature[G]{$\Delta H$}{Calor de reação \nomunit{kJ/kmol}}

As equações \ref{eq:balancodemassa} e \ref{eq:balancodeenergia}, somadas
à corrente $F$ e seu cálculo de ELV, permitem determinar a fração vaporizada,
a composição das fases e respectivas propriedades físicas em cada célula $n$.

É importante notar ainda que, da forma como as \autoref{eq:balancodemassa} e
\autoref{eq:balancodeenergia} estão escritas, massa e energia contidas em cada
célula são grandezas extensivas. 

\section{Cinética das Reações} \label{sec:cineticadasreacoes}

As reações de hidrogenação estão mostradas na
\autoref{tab:parametroscineticos1}.
Os parâmetros cinéticos e entalpias das reações estão listados na
\autoref{tab:parametroscineticos2}. Os parâmetros das reações utilizados aqui
são os mesmos utilizados por \citeonline{Rojas2014a} em seus cálculos.

\begin{table}[!htb]
\begin{center}
\caption{Reações \cite{Rojas2014a}.}
\label{tab:parametroscineticos1}
\small
\begin{tabular}{cl}
{$Identificador j$} & {Reação} \\
\hline
1 & trans-1,3-Pentadieno + $\ce{H2} \rightarrow$ trans-2-Penteno \\
2 & trans-2-Penteno + $\ce{H2} \rightarrow$ n-Pentano \\
3 & 1,3-Ciclopentadieno + $\ce{H2} \rightarrow$ Ciclopenteno \\
4 & Ciclopenteno + $\ce{H2} \rightarrow$ Ciclopentano \\
5 & Metil-1,3-Ciclopentadieno + $\ce{H2} \rightarrow$ Metilciclopenteno \\
6 & Metilciclopenteno + $\ce{H2} \rightarrow$ Metilciclopentano \\
7 & Estireno + $\ce{H2} \rightarrow$ Etilbenzeno \\
8 & Metilestireno + $\ce{H2} \rightarrow$ 1-Metil-3-Etilbenzeno \\
9 & Diciclopentadieno + $\ce{H2} \rightarrow$ Dihidrodiciclopentadieno \\
\bottomrule
\end{tabular}
\end{center}
\end{table} 

\begin{table}[!htb]
\begin{center}
\caption{Parâmetros cinéticos e entalpias de reação \cite{Rojas2014a}.}
\label{tab:parametroscineticos2}
\small
\begin{tabular}{cccc}
{$j$} & {$k^{'}_{j}$} & {$\Delta H_{j}$} & {$Ea_{j}$}
\\
{} & {$(\SI{417}{K},\SI{4,3}{MPa})(\si{h^{-1}})$} & {$(\SI{298}{K})
(\si{kJ/kmol})$} & {$(\si{kJ/kmol})$} \\
\hline
1 & \num{5.08} & \num{-1.058e5} & \num{24409} \\
2 & \num{0.32} & \num{-1.155e5} & \num{35370} \\
3 & \num{12,0} & \num{-1.016e5} & \num{14947} \\
4 & \num{0,80} & \num{-1.102e5} & \num{24409} \\
5 & \num{5.08} & \num{-1.038e5} & \num{24409} \\
6 & \num{0.32} & \num{-1.013e5} & \num{35370} \\
7 & \num{8.84} & \num{-1.158e5} & \num{14947} \\
8 & \num{5.08} & \num{-1.161e5} & \num{24409} \\
9 & \num{5.08} & \num{-1.161e5} & \num{14947} \\
\bottomrule
\end{tabular}
\end{center}
\end{table} 

Conforme já explicado, o sistema reacional foi considerado como sendo um
conjunto de hidrogenações irreversíveis e independentes da concentração de
hidrogênio. A constante de pseudo-primeira ordem da reação $j$ na célula $n$,
$k^{'}_{j,n}$, é definida como segue:
\begin{equation}
k^{'}_{j,n} = \eta_ik_jC^{*}_{\ce{H2}}
\label{eq:constantepseudoprimeiraordem}
\end{equation}
sendo $\eta$ o fator de efetividade de intradifusão, $k$ a
constante de taxa de reação instrínseca e $C^{*}_{\ce{H2}}$ a concentração de
equilíbrio de hidrogênio em fase líquida.

\nomenclature{$k^{'}$}{Constante de taxa de reação de pseudo-primeira
ordem \nomunit{kg/(kg_{cat} h)}}
\nomenclature{$k$}{Constante de taxa de reação intrínseca
\nomunit{kg m^3/(kg_{cat} h kmol_{H2})}}
\nomenclature{$C^{*}_{H_2}$}{Concentração de equilíbrio de hidrogênio em fase
líquida \nomunit{kmol/m^3}}
\nomenclature[G]{$\eta$}{Fator de efetividade de intradifusão}

A constante da taxa de reação específica $k^{*}$ é calculada pela
equação de van't Hoff, na $T^{ref}$ = \SI{417}{K}, como segue:
\begin{equation}
k^{*}_{j,n} = \dfrac{k^{'}_{j,n}} {\rho^{L}_{n+1}} \exp
\left[{\dfrac{-Ea_j}{R} \left (\dfrac{1}{T_{n+1}} -
\dfrac{1}{T^{ref}} \right )}\right]
\label{eq:constantetaxareacaoespecifica}
\end{equation}
sendo $\rho^L$ a massa específica da fase líquida.

Portanto, a taxa $r$ da reação $j$ na célula $n$ é apresentada da seguinte
forma:
\begin{equation}
r_{j,n} = k^{*}_{j,n}C^{S}_{i,n}
\label{eq:taxareacao}
\end{equation}
onde o vetor $C^{S}$ representa a concentração das espécies químicas na
superfície da fase sólida. Essa concentração é determinada pela transferência de
massa líquido-sólido, como está apresentado na \autoref{sec:interfaceliquidosolido}.

Nota-se na equação \autoref{eq:constantetaxareacaoespecifica} que foram usadas
temperatura e massa específica da corrente de saída da célula ($n+1$).
Essa opção foi feita de forma a que a premissa de que cada célula
é um CSTR (mistura perfeita) é respeitada.
De qualquer maneira, para um número grande de células, a utilização das propriedades
avaliadas na célula $n$ ou na célula $n+1$ traria pouca diferença aos resultados
finais.

\nomenclature{$k^{*}$}{Constante da taxa de reação específica
\nomunit{kmol/(m^3/(kg_{cat} h)}}
\nomenclature[R]{$L$}{Designação da fase líquida}
\nomenclature[R]{$S$}{Designação da fase sólida}
\nomenclature{$T^{ref}$}{Temperatura de referência \nomunit{K}}
\nomenclature[G]{$\rho$}{Massa específica \nomunit{kg/m^3}}
\nomenclature{$T$}{Temperatura \nomunit{K}}
\nomenclature{$R$}{Constante universal dos gases \nomunit{kJ/(kmol K)}}
\nomenclature{$Ea$}{Energia de ativação \nomunit{kJ/kmol}}
\nomenclature[S]{$B$}{Leito catalítico}
\nomenclature{$C$}{Concentração \nomunit{kmol/m^3}}

\section{Equilíbrio de Fases} \label{sec:termodinamica}

A abordagem termodinâmica utilizada para prever o ELV foi a $\phi_i$ -
$\phi_i$, a mesma utilizada por \citeonline{Rojas2014a}, com a EoS de SRK
\cite{Soave1972}, e seguindo as recomendações e os parâmetros do trabalho feito
por \citeonline{Zhou2006} para a solubilidade de hidrogênio em gasolina de
pirólise. Esta recomendação consiste em utilizar a regra de mistura clássica de
van der Waals \cite{VanderWaals1873}, com parâmetros de interação binária
específicos para cada um dos pares de hidrogênio/hidrocarboneto. Para o caso da
regra de mistura escolhida, os parâmetros cruzados $a_{ij}$ estão definidos
da seguinte forma\cite{Peng1976,Soave1972}:
\begin{equation}
a_{ij} = \sqrt{a_ia_j}(1-\delta_{ij})
\label{eq:parametroaij}
\end{equation}
onde o parâmetro de interação binária $\delta_{ij}$ sugerido por
\apud{Gray1985}{Rojas2014} é definido por:
\begin{equation}
\delta_{ij} = \num{0,0067}+\dfrac{\num{0,63375}S^3}{1+S^3}
\label{eq:deltaij}
\end{equation}
e
\begin{equation}
S = \dfrac{T_{c,j}-\num{50}}{\num{1000}-T_{c,j}}
\label{eq:deltaij}
\end{equation}

Para a avaliação do ELV e demais propriedades, foi utilizado o pacote
termodinâmico do simulador de processos iiSE (\emph{Industrial Integrated
Simulation Environment}). Para tanto, foi criada uma simulação no iiSE, onde
foram colocadas as composições da carga e da corrente de \emph{quench}. O \emso,
por sua vez, utiliza a simulação criada em iiSE para os cálculos termodinâmicos,
e somente para este fim. Vale aqui explicar que o índice $j$ na
\autoref{eq:deltaij} refere-se especialmente a componente, e não a reação
química.

Para finalizar esta seção, vale esclarecer que, da forma como foi implementada
no \emso, a corrente $F$ não só contém a informação de vazão e composição molar
(total e por componente), mas também possui uma rotina de cálculo de
\emph{flash}, utilizando como variáveis de entrada pressão, temperatura e
composição global. Dessa forma, ficam incluídos os efeitos de dissolução de
hidrogênio na fase líquida bem como a vaporização da fase líquida.

\nomenclature[G]{$\phi$}{Coeficiente de fugacidade}
\nomenclature[G]{$\delta_{ij}$}{Parâmetro de interação binária}
\nomenclature[Z]{iiSE}{\emph{Industrial Integrated Simulation Environment}}
\nomenclature{$a_{ij}$}{Parâmetro cruzado entre as espécies $i$ e $j$}
\nomenclature{$a$}{Parâmetro de atração}

\section{Transferência de Massa na Interface Líquido-Sólido}
\label{sec:interfaceliquidosolido}

Como a transferência de massa gás-líquido foi desprezada, compete aqui
apresentar as equações para determinar a concentração das espécies químicas
na superfície das partículas de catalisador, $C^S$. 

A primeira dessas equações é fruto de um balanço de massa, onde a
transferência de massa na interface líquido-sólido de um reagente é igual à
velocidade com que ele é consumido na reação:
\begin{equation}
k_{i,n}^{LS}a^{LS}(C^L_{i,n}-C^S_{i,n}) = \rho_B
\displaystyle\sum_{j=1}^{N_{reac}}
\nu_{i,j}r_{j,n}
\label{eq:transferenciamassa}
\end{equation}
sendo $C^{L}$ a concentração de qualquer espécie química presenta na fase
líquida e $k^{LS}$ o coeficiente de transferência de massa na interface líquido
sólido, calculado para cada reagente.

\nomenclature{$k^{LS}$}{Coeficiente de transferência de massa líquido-sólido
\nomunit{m/h}}

O parâmetro $a^{LS}$ é a área específica do catalisador é função do diâmetro
das partículas $d_p$ de catalisador e da porosidade do leito $\epsilon_B$:
\begin{equation}
a^{LS} = 6 \dfrac{(1-\epsilon_B)}{d_p}
\label{eq:aLS}
\end{equation}

\nomenclature{$a^{LS}$}{Área suferficial específica do catalisador
\nomunit{m^2/m^3}}

Para determinar $k_{LS}$, foi utilizada a equação de \citeonline{Hirose1976}
para o número de Sherwood dos componentes em fase líquida ($Sh^L$):
\begin{equation}
\epsilon_BSh^L_{i,n} = \num{0,8}(Re^L_n)^{\num{0,5}}(Sc^L_{i,n})^{1/3} \qquad
(Re<200)
\label{eq:Sh1}
\end{equation}
\begin{equation}
\epsilon_BSh^L_{i,n} = \num{0,53}(Re^L_n)^{\num{0,58}}(Sc^L_{i,n})^{1/3} \qquad
(Re>200)
\label{eq:Sh2}
\end{equation}
sendo que
\begin{equation}
k_{LS,n} = \dfrac{Sh^L_{i,n}D^L_{i}}{d_p}
\label{eq:kLS}
\end{equation}
\begin{equation} 
Sc^L_{i,n} = \dfrac{\mu^L_{n}}{\rho^L_{n+1}D^L_{i,n}}
\label{eq:Sc}
\end{equation}
$Re^L$ é o número de Reynolds da fase líquida, $Sc$ é o número de Schmidt,
$D^L_{i}$ é difusividade molecular do composto $i$ na fase líquida e $\mu^L$ é
viscosidade da fase líquida.

\nomenclature{$Sh$}{Número de Sherwood}
\nomenclature{$Re$}{Número de Reynolds}
\nomenclature{$Sc$}{Número de Schmidt}
\nomenclature{$D$}{Difusividade molecular \nomunit{m^2/s}}
\nomenclature[G]{$\mu$}{Viscosidade \nomunit{cP}}

\section{Parâmetros Hidrodinâmicos} \label{sec:hidrodinamica3}

Essa seção tem por objetivo apresentar as principais equações utilizadas no
cálculo de parâmetros hidrodinâmicos.

Para o cálculo da perda de carga, \citeonline{Rojas2014a} utilizaram uma equação
do tipo Ergun \apud{Ergun1952}{Holub1993}, com correções e constantes propostas
por \citeonline{Benkrid1997}. Essa equação é válida para regimes de alta
interação (borbulhamento, conforme a premissa adotada). Para o presente
trabalho, portanto, adotou-se a mesma equação, que é mostrada a seguir na forma
discretizada:
\begin{equation}
\label{eq:deltaP}
\begin{split}
\dfrac{\Delta P_{n}}{L/N_{Disc}} = \dfrac{1}{\epsilon_B^{3}}
\left(\dfrac{\dfrac{u^{G}_{n}}{u^{L}_{n}} +
1}{\num{0,49}\dfrac{u^{G}_{n}}{u^{L}_{n}} + 1}\right)
\left[\dfrac{150}{36}\left(\dfrac{6(1-\epsilon_B)} {d_P}+\dfrac{4}{D_R}
\right)^2\mu^{L}_{n}u^{L}_{n} + \dfrac{\num{1,75}}{6}
\left(\dfrac{6(1-\epsilon_B)} {d_P} + \dfrac{4}{D_R}\right)
\rho^{L}_{n}(u^{L}_{n})^2 \right]
\end{split}
\end{equation}
onde $u$ é a velocidade da fase e $L_n$ é dado por $L/N_{Disc}$ de cada leito.

\nomenclature{$u$}{Velocidade superficial da fase \nomunit{m/s}}
\nomenclature[S]{$R$}{Reator}

Visto que a pressão de operação do reator é muito acima da pressão atmosféricas,
a equação usada para estimar a retenção de líquido $\epsilon^L$ foi a proposta
por \citeonline{Larachi1991}, que é uma dentre outras
equações levantadas para sistemas de alta pressão \cite{Ancheyta2011}.
\begin{equation}
log \left (1-\dfrac{\epsilon_{n}^L}{\epsilon_B} \right) =
-\dfrac{\num{1,22}(We_{n}^L)^{\num{0,15}}}{(Re_{n}^L)^{\num{0,20}}(X_{n}^G)^{\num{0,15}}}
\label{eq:epsilonL}
\end{equation}
onde $We$ é o número de Webber e $X^G$ é o número de Lockhart-Martinelli
para a fase gás. Ambos são definidos a seguir.
\begin{equation}
We_{n}^L = \dfrac{(u_{n}^L)^2d_p\rho_{n}^L}{\sigma_{n}^L}
\label{eq:webber}
\end{equation}
\begin{equation}
X_{n}^G = \dfrac{u_{n}^G}{u_n^L} \sqrt{\dfrac{\rho^L}{\rho^G}}
\label{eq:X}
\end{equation}
sendo que $\sigma^L$ é a tensão superficial da fase líquida.

\nomenclature[G]{$\sigma$}{Tensão superficial \nomunit{N/m}}
\nomenclature{$We$}{Número de Webber}
\nomenclature{$X$}{Número de Lockhart-Martinelli}

Para verificar a premissa de que o molhamento do catalisador é completo
($\eta_{CE} = 1$), será utilizada a equação proposta por
\citeonline{Al-Dahhan1995} para sistemas de alta pressão, como segue:
\begin{equation}
\eta_{CE,n} = \num{1,104}(Re_n^L)^{1/3} \left [
\dfrac{1 + [(\Delta P_n/L_{n})/(\rho_{n}^L g)]}{Ga_{n}^{L}} \right ]
\label{eq:molhamento}
\end{equation}
sendo $Ga$ o número de Galileu e $g$ a aceleração da gravidade.

\nomenclature{$Ga$}{Número de Galileu}
\nomenclature{$g$}{Aceleração da gravidade \nomunit{m/s^2}}

O número Peclet (neste caso, $Bo$) é o parâmetro utilizado na literatura para
verificar se na modelagem do reator o fenômeno de dispersão axial é relevante ou não.
\begin{equation}
Bo_n^L = \dfrac{L_n}{d_p}
\num{2,3}(Re_n^L)^{\num{0,33}}(Ga_{n}^{L})^{\num{-0,19}}
\label{eq:numerodepeclet}
\end{equation}

A \autoref{eq:numerodepeclet} apresenta a correlação para o número de
Peclet utilizada neste trabalho, e que foi proposta por
\citeonline{Cassanello1992}.

\nomenclature{$Pe$}{Número de Peclet}

\section{Implementação do Modelo} \label{sec:implementacao}

Para a implementação do modelo foram usados dois softwares: iiSE e \emso. 

\subsection{iiSE} \label{sec:iise}

O iiSE é uma ferramenta de simulação desenvolvida pela empresa VRTech para a
simulação de processos químicos e petroquímicos. iiSE possibilita a
montagem das simulações de maneira gráfica e também permite a comunicação
com o Microsoft Excel. Há também a possibilidade de exportar os resultados
de alguns equipamentos no formato .mso para serem usando no simulador \emso.

Como já explicado na \autoref{sec:termodinamica}, o iiSE serviu como uma
ferramenta para o cálculo de ELV das correntes $F$, dados $T_n$, $P_n$ e
composição. Nele não foram implementadas quaisquer equações de balanço ou
correlações; ele apenas é chamado pelo \emso para solucionar os cálculos de
\emph{flash} necessários.

Ao pacote termodinâmico do iiSE foram adicionadas as constantes da equação de
SRK levantadas por \citeonline{Zhou2006} para a solubilidade de hidrogênio em
gasolina de pirólise.

\subsection{EMSO} \label{sec:EMSO}

O simulador \emso{} é uma ferramenta para modelagem, simulação e otimização de
sistemas, com foco principal em respostas dinâmicas. O \emso{} realiza a
verificação da consistência das unidades de medida, da resolubilidade do sistema
de equações e das condições iniciais. As três entidades principais dessa
linguagem de modelagem são: modelos (\code{Model}), equipamentos
(\code{Device}) e fluxograma (\code{FlowSheet}) \cite{Soares2003}.

\code{Model}s são descrições matemáticas de um tipo de equipamento (bomba,
reator, corrente de fluxo); um \textit{equipamento} é uma instância de um
\textit{modelo}; e o \textit{fluxograma} representa o processo a ser analisado,
que é composto por um conjunto de \textit{equipamentos} \cite{Soares2003}.

O \emso{} já possui uma biblioteca de modelos prontos para o uso (EML -
\emph{\emso{} Model Library}). Para a solução do presente trabalho, dois modelos
extras foram criados.

O primeiro modelo representa as correntes de fluxo que entram e saem de
cada célula, incluindo o cálculo de \emph{flash}. Esse modelo foi chamado de
\code{streamTP} e baseou-se nos modelos já criados para corrente na EML, com
a peculiaridade de receber como parâmetros de entrada temperatura, pressão e
composição. O código do modelo \code{streamTP} está no
\autoref{chap:streamTP}.

O segundo modelo (\autoref{chap:modeloleitofixo}) foi o de um reator de leito
fixo. Nele é possível utilizar tanto fluxo monofásico quanto fluxo bifásico.
Esse modelo contém a declaração das principais variáveis, além das principais
equações (balanços de massa e energia, por exemplo).

Por fim, para unir todos os modelos e inserir equações e variáveis
específicas para a finalidade do presente trabalho, foi criado o fluxograma do
processo, que está disponível no \autoref{chap:fluxogramaprocesso}.

\nomenclature[Z]{EML}{\emso{} Model Library}

