%
% 
%
\chapter{Introdução} \label{chap:introducao}
%\emph{Isto é uma sinopse de capítulo. A ABNT não traz nenhuma normatização a
%respeito desse tipo de resumo, que é mais comum em romances e livros técnicos.
%}

Os catalisadores têm sido aplicados pela humanidade desde a antiquidade
em atividades tais como produção de vinho, pão e queijo. Mas foi em 1835 que
Berzelius reuniu obeservações feitas por químicos, sugerindo que uma pequena
quantidade de uma substância de origem externa afetavam grandemente o curso das
reações químicas. A essa substância possuidora de tal força misteriosa Berzelius
chamou de catalítica. Em 1894, Ostwald propôs que catalisadores são
substâncias que aceleram a taxa de uma reação química, sem serem consumidos
durante a reação \cite{Oyama1988}.

Os principais catalisadores aplicados na indústria são os catalisadores
sólidos. A descoberta de catalisadores sólidos e suas aplicações a processos
químicos no início do século XX causou grandes avanços na indústria química. A
maior parte dos processos catalíticos empregam reatores de leito fixo. Na
indústria petroquímica, reatores catalíticos de leito fixo são usados para
produção de óxido de etileno, butadieno, anidrido maleico, anidrido ftálico,
estireno, entre outros produtos. Já na indústria do petróleo, destacam-se:
reforma catalítica, isomerização, hidrodessulfurização e hidrocraqueamento \cite{Froment2011}.

Dentre os reatores de leito fixo estão os reatores de leito gotejante
(TBR - \emph{Trickle Bed Reactor}), que nada mais são do que um tipo de reator
catalítico multifásico de leito empacotado (PBR - \emph{Packed Bed Reactor}).
Dentre as diversas aplicações dos reatores tipo PBR está a hidrogenação
seletiva de gasolina de pirólise. A gasolina de pirólise (\emph{PYGAS}) é um dos
produtos obtidos no processo de craqueamento a vapor quando da produção de
olefinas \cite{Cheng1986}. Sua composição instável a impede de ser utilizada sem
processos de refinamento adequados \cite{Derrien1986}.

Ao longo dos anos foram desenvolvidos diversos modelos matemáticos para
entender, projetar, simular ou otimizar o desempenho de reatores, especialmente
os reatores de leito fixo. Avanços recentes em recursos computacionais e
técnicas núméricas permitem a simulação do escoamento bifásido gás-líquido no
interior dos TBRs\cite{Ranade2011}.

Com a hidrogenação seletiva de gasolina de pirólise não
foi diferente. Vários autores se preocuparam em modelar esse
processo, adotando premissas e simplificações até hoje consideradas.
Recentemente, \citeonline{Rojas2014a} apresentaram uma modelagem de
um reator de hidrogenação seletiva de gasolina de pirólise que aplica novos
parâmetros termodinâmicos para a solubilidade de hidrogênio \cite{Rojas2014}. 

Este trabalho tem por objetivo geral o desenvolvimento de um modelo
matemático genérico para reatores de leito fixo, bifásicos ou trifásicos,
aplicando a abordagem de modelagem por rede de células. Com essa
abordagem, espera-se incluir o equilíbrio líquido-vapor comumente
desconsiderado por simplificação. 

Como objetivo específico, pretende-se apresentar aqui a modelagem e a
simulação de um reator de hidrogenação seletiva de gasolina de pirólise,
utilizando dados publicados por \citeonline{Rojas2014a}. Além disso, para o
reator estudado neste trabalho, serão apresentadas algumas respostas dinâmicas
que refletem situações típicas da indústria.

Para alcançar os objetivos propostos, será utilizado o simulador genérico de
processos \emso{} \cite{Soares2003} e seu ambiente de desenvolvimento de
modelos. O modelo de reator de leito fixo gerado neste estudo fará parte da
biblioteca EML (EMSO Model Library). A EML é distribuída no conceito de
\emph{software} livre, disponibilizando todos os modelos via internet e sem custo.

Esta dissertação está dividida em cinco capítulos, como segue:

O \autoref{chap:introducao} (este capítulo) trata da introdução e do objetivo
deste trabalho. No \autoref{chap:revisaobibliografica} está a pesquisa
de referências realizada que, inicialmente, trata de maneira breve sobre a
gasolina de pirólise e sua produção. Segue no mesmo capítulo a descrição da
modelagem de reatores de leito gotejante encontrada na literatura e, finalmente,
há a descrição de alguns aspectos da modelagem de reatores de hidrogenação de
gasolina de pirólise. O \autoref{chap:moddesenvolvidos} descreve o processo de
hidrogenação de gasolina de pirólise e a modelagem matemática desenvolvida deste
processo neste trabalho.

Os resultados da simulação do estado estacionário e de algumas
respostas dinâmicas estão no \autoref{chap:resultados}. Por fim, no
\autoref{chap:conclusoes} são apresentadas as conclusões do trabalho bem como
sugestões para trabalhos posteriores.
