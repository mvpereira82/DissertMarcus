%
% 
%
\chapter{Introdução} \label{chap:introduction}

%\emph{Isto é uma sinopse de capítulo. A ABNT não traz nenhuma normatização a
%respeito desse tipo de resumo, que é mais comum em romances e livros técnicos.
%}

Os catalisadores tem sido aplicados pela humanidade desde a antiquidade
em atividades tais como produção de vinho, pão e queijo. Em muitos
casos, percebeu-se que a adição de uma pequena quantidade da batelada anterior
para que fosse feita a batelada atual. Mas foi em 1835 que Berzelius reuniu
obeservações feitas por químicos, sugerindo que uma pequena quantidade de
uma substância de origem externa afetavam grandemente o
curso das reações químicas. A essa substância possuidora de tal força
misteriosa Berzelius chamou de catalítica. Em 1894, Ostwald propôs que
catalisdores são substâncias que aceleram a taxa de uma reação química, sem
serem consumidos durante a reação \cite{Oyama1988}.

A descoberta de catalisadores sólidos e suas aplicações a processos químicos no
início do século 20 causou grandes avanços na industria química. A maior parte
dos processos catalíticos consiste em reatores de leito fixo. Na industria
petroquímica, reatores catalíticos de leito fixo são usados para produção de
oxido de etileno, butadieno, anidrido maleico, anidrido ftálico, estireno,
entre outros produtos. Já na industria do petróleo, destacam-se as aplicações:
reforma catalítica, isomerização, hidrodessulfurização e hidrocraqueamento
\cite{Froment2011}.

Dentre os reatores de leito fixo estão os reatores de leito gotejante
(\emph{Trickle bed reactors}). Os reatores de leito gotejante são
equipamentos de contato gás-líquido-sólido e compreendem uma família de
reatores nos quais a fase líquida se move em um escoamento descendente, no
sendito da força da gravidade, enquanto o gás pode escoar tanto ascendentemente
(escoamento contracorrente) quanto descendentemente (escoamento cocorrente)
\cite{Ranade2011}. Uma das aplicações de reatores de leito gotejante é
hidrogenação de gasolina de pirólise. A gasolina de pirólise é um dos produtos obtidos no processo de
craqueamento a vapor quando da produção de olefinas \cite{Cheng1986}.

Ao longo dos anos foram desenvolvidos diversos modelos matemáticos para
entender, projetar, simular ou otimizar o desempenho de reatores, especialmente
os reatores de leito fixo. Os reatores de leito gotejante figuram entre os
reatores de leito fixo que mais exigem capacidade computacional em suas
simulações. Recentemente avanços têm sido realizados com o aumento da capacidade
computacional disponível nas industrias e universidades.

Este trabalho tem por objetivo geral o desenvolvimento de um modelo
matemático genérico para reatores de leito fixo, bifásico ou trifásicos,
aplicando a abordagem de modelagem por rede de células. Espera-se que, com a modelagem por
rede de células, as simulações de reatores de leito fixo sejam mais completas,
i.e., que aceitem a incorporação de fenômenos comumente desconsiderados.

Como objetivo específico, pretende-se apresentar aqui a modelagem e a
simulação de um reator de hidrogenação seletiva de gasolina de pirólise,
utilizando dados publicados na literatura. Além disso, para o reator estudado
neste trabalho, serão apresentadas algumas respostas dinâmicas, considerando
algumas situações típicas.

Para realizar os objetivos propostos, será utilizado o simulador genérico de
processos \emso\ \cite{Soares2003} e seu ambiente de desenvolvimento de
modelos. O modelo de reator de leito fixo gerado neste estudo fará parte da
biblioteca EML (EMSO Model Library). A EML é distribuída no conceito de \emph{software} livre,
disponibilizando todos os modelos via internet e sem custo.

Esta dissertação está dividida em seis capítulos, como segue:

O \autoref{chap:introduction} (este capítulo) trata da introdução e do objetivo deste trabalho.

O \autoref{chap:revisaobibliografica} apresenta a pesquisa de referências
realizada, tratando brevemente da gasolina de pirólise, dando mais ênfase nos
reatores de leito fixo, apresentando as diferentes abordagens de modelagem
já utilizadas.

O \autoref{chap:moddesenvolvidos} descreve o processo de
hidrogenação de gasolina de pirólise, a modelagem matemática
desenvolvida deste processoo processo e, finalmente, a implementação
deste modelo em \emso\ .

Os resultados da simulação do estado estacionário e de algumas
respostas dinâmicas estão no \autoref{chap:resultados}.

Finalmente, no \autoref{chap:conclusoes}, são apresentadas as conclusões do
trabalho, assim como sugestões para trabalhos posteriores.
