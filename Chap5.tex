%
% 
%
\chapter{Conclusões e Sugestões} \label{chap:conclusoes}
%

A modelagem e simulação de reatores trifásicos representa um grande
desafio para simuladores de processo. Muitos estudos academicos tem
sido desenvolvidos com o objetivo de descrever e prever o comportamento desses
equipamentos, quer seja em seu estado estacionários, quer seja as respostas
dinâmicas. E várias são as abordagens aplicadas para tentar melhor refletir os
fenômenos envolvidos nos reatores trifásicos, ponderando sempre o custo
computacional de cada abordagem.

A aplicação da modelagem por células no reator estudado neste trabalho permitiu
conhecer melhor o comportamento das propriedades termodinâmicas e hidrodinâmicas
das fases líquida e gasosa ao longo dos leitos. Além disso, o modelo aqui
apresentado mostrou uma evolução à modelagem feita por \citeonline{Rojas2014a},
já que algumas simplificações foram removidas. Além disso,
diferentemente de \citeonline{Rojas2014a}, os balanços de massa e energia foram
incrementados com a energia envolvida na mudança de fase dos componentes. Para
modelagens dessa natureza (por células), o \emso{} se mostrou uma ferramenta
poderosa, prática e de rápida convergência.

A utilização da ferramenta \emso{} para a modelagem por células mostrou-se
ainda mais vantajosa ao permitir a avaliação do comportamento dinâmico do reator
em algumas situações hipotéticas, mas que são bem comuns na industria.

É bem verdade que a utilização de uma vazão de carga do reator superior à
reportada no trabalho de \citeonline{Rojas2014a}, necessária para
tornar o processo coerente, prejudicou melhores avaliações nas comparações
apresentadas. Mesmo assim, pode-se dizer que este trabalho cumpriu
com seus objetivos, quais sejam:

\begin{enumerate}
  \item Apresentar o desenvolvimento de um modelo matemático genérico para
  reatores de leito fixo, bifásico ou trifásicos, aplicando a abordagem de
  modelagem por rede de células; e 
  \item Mostrar modelagem e a simulação de um reator de hidrogenação seletiva
  de gasolina de pirólise, utilizando dados publicados por
  \citeonline{Rojas2014a}.
\end{enumerate}

O aspecto a ser melhorado neste trabalho é a aplicação de equações
hidrodinâmicas mais adequadas à faixa de operação do reator. De acordo
com os resultados aqui alcançados, este se apresenta operando em regime de
gotejamento, e não em regime de borbulhamento.

Considerando a flexibilidade que o \emso{} possui por ser um simulador
voltado para equações, duas sugestões para melhorar a modelagem aqui apresentada
emergem, quais sejam:

\begin{itemize}
\item {Realizar o balanço de massa separadamente para as fases}
\item {Alterar o balaço de espaços a cada célula, considerando não mais o
equilíbrio termodinâmico mas, sim, as retenções de líquido e gás}
\end{itemize}

Uma sugestão que alteraria completamente as equações aqui apresentadas seria
aplicar a modelagem baseada em estágios. Com a utilização do \emso{} atrelado ao
iiSE, essa abordagem tende a ser mais completa do que a que foi apresentada
aqui. Contudo, a desvantagem dela é a necessidade de correlações nem sempre
adequadas para determinar os coeficientes de transferência de massa.

